\documentclass[14pt, a4paper]{extreport}

% Constant
\newcommand{\CourseTitle}{Програмування комп'ютерних та віртуальних мереж}
\newcommand{\Variant}{5}
\newcommand{\StudentGroup}{ІМ-51мн}
\newcommand{\CourseNumber}{1}
\newcommand{\StudentName}{Ковальов Олександр}
\newcommand{\Teacher}{доцент, Долголенко Олександр Миколайович}
\newcommand{\Year}{2025}
% Variable
\newcommand{\LabNumber}{3}
\newcommand{\Topic}{Створення лінійної SDN мережі на базі OpenFlow}
\newcommand{\SubmissionDate}{16.10.2025}

\usepackage{caption}
\usepackage{enumitem}
\usepackage{fancyhdr}
\usepackage{float}
\usepackage{fontspec}
\usepackage{geometry}
\usepackage{graphicx}
\usepackage{hyperref}
\usepackage{hyphenat}
\usepackage{indentfirst}
\usepackage{listings}
\usepackage{polyglossia}
\usepackage{setspace}
\usepackage{unicode-math}
\usepackage{xcolor}



\defaultfontfeatures{Ligatures=TeX}
\graphicspath{{./images/}}
\fancyhf{}
\hypersetup{unicode=true, colorlinks=true, linkcolor=black, urlcolor=black}
\pagestyle{fancy}
\setlength{\headheight}{18pt}
\setmainfont{Times New Roman}
\setmathfont{Latin Modern Math}
\setmonofont[Scale=0.8]{DejaVu Sans Mono}
\setotherlanguage{english}
\setsansfont{Arial}
\onehalfspacing

\renewcommand\headrulewidth{0pt}
\cfoot{\thepage}

\geometry{
	a4paper,
	left=25mm,
	right=10mm,
	top=15mm,
	bottom=15mm,
}

\definecolor{codegreen}{rgb}{0,0.6,0}
\definecolor{codegray}{rgb}{0.5,0.5,0.5}
\definecolor{codepurple}{rgb}{0.58,0,0.82}
\definecolor{backcolour}{rgb}{ 0.976, 0.976, 0.976 }

\lstdefinestyle{mystyle}{
	backgroundcolor=\color{backcolour},   
	commentstyle=\color{codegreen},
	keywordstyle=\color{magenta},
	numberstyle=\tiny\color{codegray},
	stringstyle=\color{codepurple},
	basicstyle=\ttfamily\footnotesize,
	breakatwhitespace=false,         
	breaklines=true,                 
	captionpos=b,                    
	keepspaces=true,                 
	numbers=left,                    
	numbersep=5pt,                  
	showspaces=false,                
	showstringspaces=false,
	showtabs=false,                  
	tabsize=2
}

\lstset{style=mystyle}

\setlist[enumerate]{
	itemsep=0.0\baselineskip,
	left=1.25cm,
	rightmargin=10mm,
	labelsep=0.5cm,
	listparindent=1.25cm,
	parsep=0pt
}

\hypersetup{
	colorlinks=true,
	linkcolor=blue,
	filecolor=magenta,      
	urlcolor=cyan
}


\begin{document}
	\tolerance=350 % Or just increase the number
	\emergencystretch=3em
	
	\begin{titlepage}
		\begin{center}
			{Національний технічний університет України\\
				«Київський політехнічний інститут імені Ігоря Сікорського» \\[1.0em] }
			{Факультет інформатики та обчислювальної техніки\\}
			{Кафедра обчислювальної техніки \\[5.0em]}
			
			{\textbf{ЗВІТ}\\[1em]}
			{\textbf{з лабораторної роботи №\LabNumber} \\}
			{\textbf{з дисципліни "\CourseTitle"} \\[2.0em]}
			
			{\textbf{Тема: \Topic} \\[2.0em]}
			
			{\textbf{Варіант №\Variant} \\[5.0em]}
			
			\begin{flushright}
				Виконав: \\
				Студент \CourseNumber{} курсу, групи \StudentGroup \\
				\StudentName \\[2.0em]
			\end{flushright}
			
			\begin{flushright}
				Перевірила: \\
				\Teacher \\[2.0em]
			\end{flushright}
			
			\begin{flushright}
				Дата здачі: \SubmissionDate \\[5.0em]
			\end{flushright}
		
			\vfill
			КИЇВ -- \Year
		\end{center}
	\end{titlepage}
	
	\setlength{\parindent}{1.25cm}
	
	\textbf{Мета роботи.} Виконати автоматичне розпаралелювання арифметичного виразу.
	
	\textbf{Вхідні дані:} Коректний арифметичний вираз після успішного виконання лексичного та синтаксичного аналізу (результат виконання лабораторної роботи №1).
	
	\textbf{Завдання:} За аналітичним записом арифметичного виразу (АВ) побудувати його дерево паралельної форми максимальної ширини (максимальна кількість операцій в одному ярусі) та мінімальної довжини (мінімально можлива кількість ярусів). Для виконання цієї лабораторної роботи необхідно розробити алгоритм побудови дерева паралельної форми за записом АВ та реалізувати цей алгоритм на будь-якій мові програмування. Спосіб зображення отриманого дерева визначається студентом. Окремої уваги заслуговують фрагменти АВ типу \texttt{A/S/D/F/G/H} або \texttt{A-S-D-F-G-H}. У цих випадках для того, щоб розпаралелювання стало можливим можна використати один із двох способів:
	
	\begin{enumerate}[noitemsep, nolistsep]
		\item Можна замінити фрагмент АВ \texttt{A/S/D/F/G/H} на \texttt{A/(S*D*F*G*H)} або фрагмент \texttt{A-S-D-F-G-H} на \texttt{A-(S+D+F+G+H)};
		\item Можна частину операцій \texttt{/} замінити на \texttt{*}, а частину операцій -- відповідно на \texttt{+}. 
	\end{enumerate}
	
	В обох випадках ми можемо досягти прискорення виконання заданого АВ за рахунок додаткового паралелізму, а також зменшення загальної трудомісткості обчислень, оскільки зменшується число операцій ділення і збільшується кількість операцій множення.
	
	\begin{center}
		\textbf{Хід роботи.}
	\end{center}
	
	У ході виконання роботи було реалізовано повний конвеєр для автоматичного розпаралелювання арифметичних виразів. На відміну від першої лабораторної роботи, яка фокусувалася лише на виявленні лексичних та синтаксичних помилок, ця робота починається з коректного, синтаксично перевіреного потоку лексем. 
	
	Основою для всіх подальших трансформацій є абстрактне синтаксичне дерево (AST). Спочатку береться синтаксично коректний потік лексем (сформований \texttt{lexer.rs}) і перетворює його на деревовидну структуру. Для цього використовується \texttt{AstParser} -- парсер, реалізований за методом рекурсивного спуску. Він послідовно обробляє лексеми, дотримуючись пріоритету операцій: \texttt{parse\_logical\_or} викликає \texttt{parse\_logical\_and}, той викликає \texttt{parse\_expression} (для \texttt{+} та \texttt{-}), який, у свою чергу, викликає \texttt{parse\_term} (для \texttt{*} та \texttt{/}), і так далі аж до \texttt{parse\_primary}. Цей парсер також коректно обробляє унарні операції, групування в дужках, виклики функцій та доступ до елементів масиву, будуючи ієрархічну структуру \texttt{AstNode}. Отримане початкове дерево, яке відображає пріоритет операцій вихідного виразу, є відправною точкою для подальших етапів оптимізації, трансформації та балансування.
	
	Процес компіляції було організовано як багатоетапний конвеєр, визначений у файлі \texttt{compiler.rs}. Кожен етап виконує специфічну задачу трансформації або оптимізації дерева, передаючи результат наступному етапу.
	
	Першим кроком після парсингу є початкова оптимізація, реалізована у модулі \texttt{math.rs}. Цей етап виконує рекурсивний обхід дерева знизу вгору. Його головна задача -- обчислення констант (constant folding) та виконання базових алгебраїчних спрощень. Наприклад, вирази 2 + 2 обчислюються в 4, а A * 0 -- в 0.0. Було реалізовано правила для спрощення операцій з нулем та одиницею, такі як \texttt{A + 0} (стає A), \texttt{A * 1} (стає A), \texttt{0 / A} (стає 0.0) тощо. Також було додано потужніше правило для обробки ідентичних піддерев: вирази \texttt{A - A} спрощуються до 0.0, а \texttt{A / A} -- до 1.0. Цей етап також відповідає за виявлення семантичних помилок, таких як ділення на нуль, включаючи складні випадки, як \texttt{(A*0) / (B-B)}. Якщо вираз повністю обчислюється до єдиного числа, подальша обробка припиняється.
	
	Демонстрація згортання констант:
	
	\begin{figure}[H]
		\centering
		\includegraphics[height=11cm]{01} 
	\end{figure}
	
	Наступним і ключовим етапом є трансформація дерева, реалізована в \texttt{transform.rs}. Мета цього модуля -- позбутися операцій, які заважають розпаралелюванню (бінарні віднімання та ділення) і підготувати дерево до балансування. Це досягається шляхом заміни цих операцій на їхні асоціативні еквіваленти. Було реалізовано два основних правила. По-перше, операція віднімання \texttt{A - B} перетворюється на операцію додавання з унарним мінусом: \texttt{A + (-B)}. По-друге, операція ділення \texttt{A / B} перетворюється на множення: \texttt{A * (1.0 / B)}.
	
	Найскладнішою частиною етапу трансформації є коректна обробка дужок, особливо тих, перед якими стоїть знак мінус. Це реалізовано через рекурсивну логіку для унарної операції "мінус". Коли \texttt{transform\_recursive} зустрічає унарний мінус, він аналізує вираз під ним. Якщо це інший унарний мінус, спрацьовує правило \texttt{-(-A) -> A}, і подвійний мінус скорочується. Якщо це бінарна операція, правила розкриття дужок застосовуються рекурсивно: \texttt{-(A + B)} перетворюється на \texttt{(-A) + (-B)}, а \texttt{-(A - B)} -- на \texttt{(-A) + B}. Для запуску цього механізму, сама трансформація \texttt{A - B} рекурсивно викликає \texttt{transform\_recursive} на своєму результаті \texttt{A + (-B)}. Також було додано оптимізацію для "проштовхування" мінуса до констант: \texttt{-(Num * B)} перетворюється на \texttt{(-Num) * B}. Щоб уникнути нескінченної рекурсії або подвійної трансформації (що було виявлено під час тестування), до правила ділення було додано "ідемпотентну" перевірку: якщо вираз вже має вигляд \texttt{1.0 / B}, він не трансформується повторно.
	
	Демонстрація операції трансформації:
	
	\begin{figure}[H]
		\centering
		\includegraphics[height=10cm]{02} 
	\end{figure}
	
	Після трансформації дерево може містити нові константні вирази (наприклад, \texttt{A / 2.0} стало \texttt{A * 0.5}, або \texttt{B / 1.0} стало \texttt{B * 1.0}). Тому конвеєр запускає другий етап оптимізації згортання констант. Цей прохід, використовуючи ту саму логіку з \texttt{math.rs}, "підчищає" дерево, спрощую чи \texttt{B * 1.0} до \texttt{B} і об'єднуючи нові константи, які могли опинитися поруч.
	
	Останнім основним етапом є балансування дерева, реалізоване в \texttt{balancer.rs}. Цей модуль приймає трансформоване і спрощене дерево (яке тепер складається переважно з ланцюжків асоціативних операцій \texttt{+} та \texttt{*}) і будує з нього дерево мінімальної висоти. Алгоритм працює у два кроки. Спочатку функція \texttt{collect\_operands} рекурсивно "сплощує" ланцюжки однакових асоціативних операцій (наприклад, дерево \texttt{((a + (-b)) + c)} стає списком \texttt{[a, (-b), c]}). Потім функція \texttt{build\_balanced\_tree} будує збалансоване дерево з цього списку. Вона використовує \texttt{VecDeque} як чергу: всі операнди додаються до черги. Далі в циклі, доки в черзі більше одного елемента, вузли попарно дістаються з початку черги, об'єднуються в новий \texttt{BinaryOperation} (наприклад, \texttt{left + right}), і цей новий батьківський вузол кладеться в кінець черги. Цей процес повторюється, створюючи яруси дерева знизу вгору, і гарантує, що кінцеве дерево матиме мінімально можливу висоту.
	
	Демонстрація балансування:
	
	\begin{figure}[H]
		\centering
		\includegraphics[height=12cm]{03} 
	\end{figure}
	
	Нарешті, конвеєр виконує третій, фінальний запуск згортання констант. Він потрібен для рідкісних випадків, коли процес балансування могло випадково поставити два обчислюваних вузли поруч, дозволяючи останнє фінальне спрощення. У результаті роботи цього конвеєра вхідний арифметичний вираз перетворюється на збалансоване, алгебраїчно спрощене дерево, повністю готове до паралельного виконання.
	
	\textbf{Висновок.}
	Під час виконання лабораторної роботи було успішно досягнуто її мети -- автоматичного розпаралелювання арифметичних виразів . Для цього було реалізовано багатоетапний конвеєр компіляції, що включає три ключові модулі.
		
	\pagebreak
	
	\begin{center}
		\textbf{Лістинг.}
	\end{center}
	
	\textbf{compiler.rs}
	
	\begin{lstlisting}[language=Rust]
		use crate::compiler::ast::tree::{AbstractSyntaxTree, AstParser};
		use crate::compiler::lexer::Lexer;
		use crate::compiler::syntax::SyntaxAnalyzer;
		
		pub fn compile(source: &str, is_pretty: bool) {
			// Lexical Analysis
			let tokens = tokenizer::tokenize(source);
			// Syntax Analysis
			let syntax_errors = SyntaxAnalyzer::new(&tokens).analyze();
			let is_syntax_analysis_successful = syntax_errors.is_empty();
			syntax::report(source, syntax_errors, is_pretty);
			if !is_syntax_analysis_successful {
				return;
			}
			
			// Making lexemes
			let lexemes_result = Lexer::new(tokens).run();
			let lexemes = match lexemes_result {
				Ok(lexemes) => {
					lexer::report_success(&lexemes);
					lexemes
				},
				Err(error) => {
					lexer::report_error(error);
					return;
				},
			};
			
			// AST Generation
			let ast_result = AstParser::new(lexemes).parse();
			let ast = match ast_result {
				Ok(ast) => {
					ast::tree::report_success(&ast);
					ast
				},
				Err(error) => {
					ast::tree::report_error(error);
					return;
				},
			};
			// AST Math Optimization, #1
			let ast = match compute_run(ast, 1) {
				Some(ast) => ast,
				None => return,
			};
			// AST Parallelization
			let ast_result = ast.transform();
			let ast = match ast_result {
				Ok(ast) => {
					ast::transform::report_success(&ast);
					ast
				},
				Err(error) => {
					ast::transform::report_error(error);
					return;
				},
			};
			// AST Math Optimization, #2
			let ast = match compute_run(ast, 2) {
				Some(ast) => ast,
				None => return,
			};
			// AST Balancing
			let ast_result = ast.balance();
			let ast = match ast_result {
				Ok(ast) => {
					ast::balancer::report_success(&ast);
					ast
				},
				Err(error) => {
					ast::balancer::report_error(error);
					return;
				},
			};
			// AST Math Optimization, #3
			let _ast = match compute_run(ast, 3) {
				Some(ast) => ast,
				None => return,
			};
		}
		
		fn compute_run(tree: AbstractSyntaxTree, number: u8) -> Option<AbstractSyntaxTree> {
			// AST Math Optimization
			let ast_result = tree.compute();
			let ast = match ast_result {
				Ok(ast) => {
					ast::math::report_success(&ast, number);
					ast
				},
				Err(error) => {
					ast::math::report_error(error, number);
					return None;
				},
			};
			if ast::math::check_finalization(&ast) {
				return None;
			}
			Some(ast)
		}
		
		pub mod ast {
			pub mod balancer;
			pub mod math;
			pub mod transform;
			pub mod tree;
		}
		pub mod lexer;
		pub mod syntax;
		pub mod tokenizer;
	\end{lstlisting}
	
	\textbf{tree.rs}
	
	\begin{lstlisting}[language=Rust]
		use crate::compiler::lexer::Lexeme;
		use colored::Colorize;
		
		#[derive(Debug, Clone, PartialEq)]
		pub struct AbstractSyntaxTree {
			pub peek: AstNode,
		}
		
		impl AbstractSyntaxTree {
			pub fn from_node(node: AstNode) -> Self {
				Self { peek: node }
			}
			
			pub fn pretty_print(&self) -> String {
				let mut tree = String::new();
				Self::print_recursive(&self.peek, &mut tree, "".to_string(), true);
				tree
			}
			
			fn print_recursive(node: &AstNode, tree: &mut String, prefix: String, is_last: bool) {
				let connector = if is_last { "└── " } else { "├── " };
				
				tree.push_str(&format!("{}{}", prefix.dimmed(), connector.dimmed()));
				
				let node_text = match node {
					AstNode::Number(n) => format!("{n:.3}").bright_blue(),
					AstNode::Identifier(s) => s.to_string().green(),
					AstNode::StringLiteral(s) => format!("\"{}\"", s).bright_magenta(),
					AstNode::UnaryOperation { operation, .. } => {
						operation.to_string().yellow().bold()
					},
					AstNode::BinaryOperation { operation, .. } => {
						operation.to_string().yellow().bold()
					},
					AstNode::FunctionCall { name, .. } => format!("{}(...)", name).cyan().bold(),
					AstNode::ArrayAccess { identifier, .. } => {
						format!("{}[...]", identifier).blue().bold()
					},
				};
				tree.push_str(&format!("{}\n", node_text));
				
				let new_prefix = prefix + if is_last { "    " } else { "│   " };
				
				match node {
					AstNode::Number(_) | AstNode::Identifier(_) | AstNode::StringLiteral(_) => {},
					
					AstNode::UnaryOperation { expression, .. } => {
						Self::print_recursive(expression, tree, new_prefix, true);
					},
					
					AstNode::BinaryOperation { left, right, .. } => {
						Self::print_recursive(left, tree, new_prefix.clone(), false);
						Self::print_recursive(right, tree, new_prefix, true);
					},
					
					AstNode::FunctionCall { arguments, .. } => {
						let arg_count = arguments.len();
						for (i, arg) in arguments.iter().enumerate() {
							let is_last_arg = i == arg_count - 1;
							Self::print_recursive(arg, tree, new_prefix.clone(), is_last_arg);
						}
					},
					
					AstNode::ArrayAccess {
						identifier: _,
						indices,
					} => {
						let dimensions = indices.len();
						for (i, index) in indices.iter().enumerate() {
							let is_last_arg = i == dimensions - 1;
							Self::print_recursive(index, tree, new_prefix.clone(), is_last_arg);
						}
					},
				}
			}
		}
		
		#[derive(Debug, Clone, PartialEq)]
		pub enum AstNode {
			Number(f64),
			Identifier(String),
			StringLiteral(String),
			UnaryOperation {
				operation: UnaryOperationKind,
				expression: Box<AstNode>,
			},
			BinaryOperation {
				operation: BinaryOperationKind,
				left: Box<AstNode>,
				right: Box<AstNode>,
			},
			FunctionCall {
				name: String,
				arguments: Vec<AstNode>,
			},
			ArrayAccess {
				identifier: String,
				indices: Vec<AstNode>,
			},
		}
		
		#[derive(Debug, Clone, PartialEq)]
		pub enum UnaryOperationKind {
			Minus,
			Not,
		}
		
		#[derive(Debug, Clone, PartialEq)]
		pub enum BinaryOperationKind {
			Plus,
			Minus,
			Multiply,
			Divide,
			Or,
			And,
		}
		
		pub struct AstParser {
			lexemes: Vec<Lexeme>,
			current_index: usize,
		}
		
		impl AstParser {
			pub fn new(lexemes: Vec<Lexeme>) -> Self {
				Self {
					lexemes,
					current_index: 0,
				}
			}
			
			pub fn parse(&mut self) -> Result<AbstractSyntaxTree, AstError> {
				let node = self.parse_logical_or()?;
				
				if self.peek().is_some()
				&& let Some(peek) = self.consume()
				{
					Err(AstError::NotExpectedLexeme(peek.clone()))
				} else {
					Ok(AbstractSyntaxTree { peek: node })
				}
			}
			
			fn parse_logical_or(&mut self) -> Result<AstNode, AstError> {
				let mut left_node = self.parse_logical_and()?;
				
				while let Some(Lexeme::Or) = self.peek()
				&& let Some(_) = self.consume()
				{
					let right_node = self.parse_logical_and()?;
					left_node = AstNode::BinaryOperation {
						operation: BinaryOperationKind::Or,
						left: Box::new(left_node),
						right: Box::new(right_node),
					};
				}
				Ok(left_node)
			}
			
			fn parse_logical_and(&mut self) -> Result<AstNode, AstError> {
				let mut left_node = self.parse_expression()?;
				
				while let Some(Lexeme::And) = self.peek()
				&& let Some(_) = self.consume()
				{
					let right_node = self.parse_expression()?;
					left_node = AstNode::BinaryOperation {
						operation: BinaryOperationKind::And,
						left: Box::new(left_node),
						right: Box::new(right_node),
					};
				}
				Ok(left_node)
			}
			
			fn parse_expression(&mut self) -> Result<AstNode, AstError> {
				let mut left_node = self.parse_term()?;
				
				while let Some(Lexeme::Plus) | Some(Lexeme::Minus) = self.peek()
				&& let Some(lexeme) = self.consume()
				{
					let operation = match lexeme {
						Lexeme::Plus => BinaryOperationKind::Plus,
						Lexeme::Minus => BinaryOperationKind::Minus,
						_ => return Err(AstError::UnreachableLexeme(lexeme.clone())),
					};
					
					let right_node = self.parse_term()?;
					
					left_node = AstNode::BinaryOperation {
						operation,
						left: Box::new(left_node),
						right: Box::new(right_node),
					};
				}
				
				Ok(left_node)
			}
			
			fn parse_term(&mut self) -> Result<AstNode, AstError> {
				let mut left_node = self.parse_unary()?;
				
				while let Some(Lexeme::Multiply) | Some(Lexeme::Divide) = self.peek()
				&& let Some(lexeme) = self.consume()
				{
					let operation = match lexeme {
						Lexeme::Multiply => BinaryOperationKind::Multiply,
						Lexeme::Divide => BinaryOperationKind::Divide,
						_ => return Err(AstError::UnreachableLexeme(lexeme.clone())),
					};
					
					let right_node = self.parse_unary()?;
					
					left_node = AstNode::BinaryOperation {
						operation,
						left: Box::new(left_node),
						right: Box::new(right_node),
					};
				}
				
				Ok(left_node)
			}
			
			fn parse_unary(&mut self) -> Result<AstNode, AstError> {
				if let Some(Lexeme::Not) | Some(Lexeme::Minus) = self.peek()
				&& let Some(lexeme) = self.consume()
				{
					let operation_kind = match lexeme {
						Lexeme::Not => UnaryOperationKind::Not,
						Lexeme::Minus => UnaryOperationKind::Minus,
						_ => return Err(AstError::UnreachableLexeme(lexeme.clone())),
					};
					
					let child_node = self.parse_unary()?;
					
					Ok(AstNode::UnaryOperation {
						operation: operation_kind,
						expression: Box::new(child_node),
					})
				} else {
					self.parse_primary()
				}
			}
			
			fn parse_primary(&mut self) -> Result<AstNode, AstError> {
				if let Some(lexeme) = self.consume() {
					match lexeme {
						Lexeme::Number(value) => Ok(AstNode::Number(value)),
						Lexeme::String(value) => {
							match (matches!(self.peek(), Some(Lexeme::Comma)))
							|| (matches!(self.peek_previous_by(2), Some(Lexeme::Comma)))
							{
								true => Ok(AstNode::StringLiteral(value.clone())),
								false => Err(AstError::StringOutsideFunction(value.clone())),
							}
						},
						
						Lexeme::LeftParenthesis => {
							let inner_node = self.parse_logical_or()?;
							
							if self.peek() == Some(&Lexeme::RightParenthesis) {
								self.consume();
								Ok(inner_node)
							} else {
								Err(AstError::ExpectedRightParenthesis)
							}
						},
						
						Lexeme::Identifier(name) => {
							if self.peek() == Some(&Lexeme::LeftParenthesis)
							&& let Some(_) = self.consume()
							{
								let function_name = name.clone();
								let mut args = Vec::new();
								
								if self.peek() != Some(&Lexeme::RightParenthesis) {
									loop {
										args.push(self.parse_logical_or()?);
										
										let peek = self.peek();
										
										if peek == Some(&Lexeme::Comma) {
											let _ = self.consume();
										} else if peek == Some(&Lexeme::RightParenthesis) {
											break;
										} else {
											return Err(match peek {
												None => AstError::NotExpectedEndOfExpression,
												Some(lexeme) => {
													AstError::ExpectedCommaOrRightParenthesis(
													lexeme.clone(),
													)
												},
											});
										}
									}
								}
								
								let _ = self.consume();
								
								Ok(AstNode::FunctionCall {
									name: function_name,
									arguments: args,
								})
							} else if self.peek() == Some(&Lexeme::LeftBracket) {
								let identifier = name.clone();
								let mut indices: Vec<AstNode> = Vec::new();
								
								loop {
									let _ = self.consume();
									let index = self.parse_logical_or()?;
									if self.peek() == Some(&Lexeme::RightBracket) {
										let _ = self.consume();
										indices.push(index);
										if self.peek() == Some(&Lexeme::LeftBracket) {
											continue;
										} else {
											break;
										}
									} else {
										return Err(AstError::ExpectedRightBracket);
									}
								}
								Ok(AstNode::ArrayAccess {
									identifier,
									indices,
								})
							} else {
								Ok(AstNode::Identifier(name.clone()))
							}
						},
						
						_ => Err(AstError::NotExpectedLexeme(lexeme.clone())),
					}
				} else {
					Err(AstError::NotExpectedEndOfExpression)
				}
			}
			
			fn consume(&mut self) -> Option<Lexeme> {
				if let Some(lexeme) = self.peek() {
					let lexeme = lexeme.clone();
					self.current_index += 1;
					return Some(lexeme);
				}
				None
			}
			
			fn peek(&self) -> Option<&Lexeme> {
				self.lexemes.get(self.current_index)
			}
			
			fn peek_previous_by(&self, by: usize) -> Option<&Lexeme> {
				self.lexemes.get(self.current_index - by)
			}
		}
		
		pub fn report_success(tree: &AbstractSyntaxTree) {
			log::warn!(
			"{} {}.",
			"Abstract-Syntax Tree generation",
			"success".bold().green()
			);
			log::info!("{}", tree.pretty_print());
		}
		
		pub fn report_error(error: AstError) {
			log::error!("{} {}", "AST error:".bold().red(), error);
		}
		
		#[derive(Debug, PartialEq)]
		pub enum AstError {
			ExpectedRightBracket,
			ExpectedRightParenthesis,
			ExpectedCommaOrRightParenthesis(Lexeme),
			NotExpectedEndOfExpression,
			NotExpectedLexeme(Lexeme),
			StringOutsideFunction(String),
			UnreachableLexeme(Lexeme),
			
			CannotBuildEmptyTree,
			FailedPopFromQueue,
			DivisionByZero(AstNode),
		}
		
		impl std::fmt::Display for AstError {
			fn fmt(&self, f: &mut std::fmt::Formatter<'_>) -> std::fmt::Result {
				let text = match self {
					Self::ExpectedCommaOrRightParenthesis(lexeme) => &format!(
					"Expected ',' or ')', but found \"{}\".",
					lexeme.display_type()
					),
					Self::ExpectedRightBracket => "Expected right bracket.",
					Self::ExpectedRightParenthesis => "Expected right parenthesis.",
					Self::NotExpectedEndOfExpression => "Not expected end of expression.",
					Self::NotExpectedLexeme(lexeme) => {
						&format!("Not expected lexeme \"{}\".", lexeme.display_type())
					},
					Self::StringOutsideFunction(string) => {
						&format!("String literal \"{}\" outside function call.", string)
					},
					Self::UnreachableLexeme(lexeme) => {
						&format!("Unreachable lexeme \"{}\".", lexeme.display_type())
					},
					
					Self::CannotBuildEmptyTree => {
						"Cannot build a balanced tree from zero operands"
					},
					Self::FailedPopFromQueue => {
						"Failed to pop node from the queue during tree construction"
					},
					Self::DivisionByZero(node) => &format!("Division by zero. Node: {:#?}", node),
				};
				
				write!(f, "{}", text)
			}
		}
		
		impl std::fmt::Display for UnaryOperationKind {
			fn fmt(&self, f: &mut std::fmt::Formatter<'_>) -> std::fmt::Result {
				match self {
					Self::Minus => write!(f, "-"),
					Self::Not => write!(f, "!"),
				}
			}
		}
		
		impl std::fmt::Display for BinaryOperationKind {
			fn fmt(&self, f: &mut std::fmt::Formatter<'_>) -> std::fmt::Result {
				match self {
					Self::Plus => write!(f, "+"),
					Self::Minus => write!(f, "-"),
					Self::Multiply => write!(f, "*"),
					Self::Divide => write!(f, "/"),
					Self::Or => write!(f, "|"),
					Self::And => write!(f, "&"),
				}
			}
		}
	\end{lstlisting}
	
	\textbf{math.rs}
	
	\begin{lstlisting}[language=Rust]
		use crate::compiler::ast::tree::{
			AbstractSyntaxTree, AstError, AstNode, BinaryOperationKind, UnaryOperationKind,
		};
		use colored::Colorize;
		
		impl AbstractSyntaxTree {
			pub fn compute(self) -> Result<AbstractSyntaxTree, AstError> {
				let computed = Self::compute_recursive(self.peek)?;
				
				Ok(Self::from_node(computed))
			}
			
			fn compute_recursive(node: AstNode) -> Result<AstNode, AstError> {
				match &node {
					AstNode::Number(_) | AstNode::Identifier(_) | AstNode::StringLiteral(_) => {
						Ok(node)
					},
					AstNode::UnaryOperation {
						operation: op,
						expression,
					} => match &op {
						UnaryOperationKind::Minus => {
							let child = Self::compute_recursive(*expression.clone())?;
							if let AstNode::Number(number) = child {
								Ok(AstNode::Number(-number))
							} else {
								Ok(node)
							}
						},
						UnaryOperationKind::Not => Ok(node),
					},
					AstNode::BinaryOperation {
						operation,
						left,
						right,
					} => match operation {
						BinaryOperationKind::Plus
						| BinaryOperationKind::Minus
						| BinaryOperationKind::Multiply
						| BinaryOperationKind::Divide => {
							let computed_left = Self::compute_recursive(*left.clone())?;
							let computed_right = Self::compute_recursive(*right.clone())?;
							
							// Case: (a + b) - (a + b) = 0
							// Or: (a + b) / (a + b) = 1
							if computed_left.eq(&computed_right) {
								match operation {
									BinaryOperationKind::Minus => {
										return Ok(AstNode::Number(0.0));
									},
									BinaryOperationKind::Divide => {
										if let AstNode::Number(number) = &computed_left
										&& *number == 0.0
										{
											// Case: (5 - 5) / (5 - 5)
											return Err(AstError::DivisionByZero(node));
										}
										return Ok(AstNode::Number(1.0));
									},
									_ => {},
								}
							}
							
							if let (AstNode::Number(left_number), AstNode::Number(right_number)) =
							(&computed_left, &computed_right)
							{
								let result = match operation {
									BinaryOperationKind::Plus => left_number + right_number,
									BinaryOperationKind::Minus => left_number - right_number,
									BinaryOperationKind::Multiply => left_number * right_number,
									BinaryOperationKind::Divide => {
										if *right_number == 0.0 {
											return Err(AstError::DivisionByZero(node));
										} else {
											left_number / right_number
										}
									},
									_ => unreachable!(),
								};
								Ok(AstNode::Number(result))
							} else if let AstNode::Number(number) = &computed_left {
								if number == &0.0 {
									if [
									BinaryOperationKind::Multiply,
									BinaryOperationKind::Divide,
									]
									.contains(operation)
									{
										return Ok(AstNode::Number(0.0));
									}
									if BinaryOperationKind::Plus == *operation {
										return Ok(computed_right);
									}
									if BinaryOperationKind::Minus == *operation {
										return Ok(AstNode::UnaryOperation {
											operation: UnaryOperationKind::Minus,
											expression: Box::new(computed_right),
										});
									}
								}
								if number == &1.0 && BinaryOperationKind::Multiply == *operation {
									return Ok(computed_right);
								}
								
								Ok(AstNode::BinaryOperation {
									operation: operation.clone(),
									left: Box::new(computed_left),
									right: Box::new(computed_right),
								})
							} else if let AstNode::Number(number) = &computed_right {
								if number == &0.0 {
									if BinaryOperationKind::Divide == *operation {
										return Err(AstError::DivisionByZero(node));
									}
									if BinaryOperationKind::Multiply == *operation {
										return Ok(AstNode::Number(0.0));
									}
									if [BinaryOperationKind::Plus, BinaryOperationKind::Minus]
									.contains(operation)
									{
										return Ok(computed_left);
									}
								}
								if number == &1.0 && BinaryOperationKind::Multiply == *operation {
									return Ok(computed_left);
								}
								
								Ok(AstNode::BinaryOperation {
									operation: operation.clone(),
									left: Box::new(computed_left),
									right: Box::new(computed_right),
								})
							} else {
								Ok(AstNode::BinaryOperation {
									operation: operation.clone(),
									left: Box::new(computed_left),
									right: Box::new(computed_right),
								})
							}
						},
						_ => Ok(node),
					},
					AstNode::FunctionCall { name, arguments } => {
						let mut computed_arguments = Vec::new();
						for arg in arguments {
							let arg = Self::compute_recursive(arg.clone())?;
							computed_arguments.push(arg);
						}
						
						Ok(AstNode::FunctionCall {
							name: name.clone(),
							arguments: computed_arguments,
						})
					},
					AstNode::ArrayAccess {
						identifier,
						indices,
					} => {
						let mut computed_indices = Vec::new();
						for index in indices {
							let index = Self::compute_recursive(index.clone())?;
							computed_indices.push(index);
						}
						Ok(AstNode::ArrayAccess {
							identifier: identifier.clone(),
							indices: computed_indices,
						})
					},
				}
			}
		}
		
		pub fn report_success(tree: &AbstractSyntaxTree, run: u8) {
			log::warn!(
			"Computing constants of Abstract-Syntax Tree (Run #{}) {}.",
			run.to_string().bright_magenta().italic(),
			"success".bold().green()
			);
			log::info!("{}", tree.pretty_print());
		}
		
		pub fn report_error(error: AstError, run: u8) {
			log::error!(
			"{} (Run #{}) {}",
			"Computing constants of Abstract-Syntax Tree:".bold().red(),
			run.to_string().bright_magenta().italic(),
			error
			);
		}
		
		pub fn check_finalization(tree: &AbstractSyntaxTree) -> bool {
			if let AstNode::Number(number) = &tree.peek {
				log::warn!(
				"{} = {}. {}.",
				"Computing solved code, result".bold().green(),
				number,
				"Further optimization is not needed".bold().red(),
				);
				return true;
			}
			false
		}
	\end{lstlisting}
	
	\textbf{transform.rs}
	
	\begin{lstlisting}[language=Rust]
		use crate::compiler::ast::tree::{
			AbstractSyntaxTree, AstError, AstNode, BinaryOperationKind, UnaryOperationKind,
		};
		use colored::Colorize;
		
		impl AbstractSyntaxTree {
			pub fn transform(self) -> Result<AbstractSyntaxTree, AstError> {
				let peek = Self::transform_recursive(self.peek)?;
				
				Ok(Self::from_node(peek))
			}
			
			pub fn transform_recursive(node: AstNode) -> Result<AstNode, AstError> {
				match &node {
					AstNode::Number(_) | AstNode::Identifier(_) | AstNode::StringLiteral(_) => {
						Ok(node)
					},
					
					AstNode::UnaryOperation {
						operation,
						expression,
					} => {
						let transformed_expression =
						Self::transform_recursive(*expression.clone())?;
						match operation {
							UnaryOperationKind::Not => Ok(AstNode::UnaryOperation {
								operation: operation.clone(),
								expression: Box::new(transformed_expression),
							}),
							
							UnaryOperationKind::Minus => {
								match &transformed_expression {
									// Rule: -(-A) => A
									AstNode::UnaryOperation {
										operation: UnaryOperationKind::Minus,
										expression: inner_expr,
									} => Ok(*inner_expr.clone()),
									
									// Rule: -(A + B) => (-A) + (-B)
									AstNode::BinaryOperation {
										operation: BinaryOperationKind::Plus,
										left,
										right,
									} => {
										let new_left = AstNode::UnaryOperation {
											operation: UnaryOperationKind::Minus,
											expression: left.clone(),
										};
										let new_right = AstNode::UnaryOperation {
											operation: UnaryOperationKind::Minus,
											expression: right.clone(),
										};
										Self::transform_recursive(AstNode::BinaryOperation {
											operation: BinaryOperationKind::Plus,
											left: Box::new(new_left),
											right: Box::new(new_right),
										})
									},
									
									// Rule: -(A - B) => (-A) + B
									AstNode::BinaryOperation {
										operation: BinaryOperationKind::Minus,
										left,
										right,
									} => {
										let new_left = AstNode::UnaryOperation {
											operation: UnaryOperationKind::Minus,
											expression: left.clone(),
										};
										Self::transform_recursive(AstNode::BinaryOperation {
											operation: BinaryOperationKind::Plus,
											left: Box::new(new_left),
											right: right.clone(),
										})
									},
									
									// Rule: -(Num * B) => -Num * B
									AstNode::BinaryOperation {
										operation: BinaryOperationKind::Multiply,
										left,
										right,
									} => {
										if let AstNode::Number(number) = *left.clone() {
											return Ok(AstNode::BinaryOperation {
												operation: BinaryOperationKind::Multiply,
												left: Box::new(AstNode::Number(-number)),
												right: Box::new(*right.clone()),
											});
										}
										
										Ok(AstNode::UnaryOperation {
											operation: UnaryOperationKind::Minus,
											expression: Box::new(transformed_expression),
										})
									},
									
									// Other cases (for example, -(A*B) or just -A)
									// just leaving them (for example -(A*B) or just -A)
									_ => Ok(AstNode::UnaryOperation {
										operation: UnaryOperationKind::Minus,
										expression: Box::new(transformed_expression),
									}),
								}
							},
						}
					},
					
					AstNode::FunctionCall { name, arguments } => {
						let mut transformed_arguments = vec![];
						for argument in arguments {
							transformed_arguments
							.push(Self::transform_recursive(argument.clone())?);
						}
						
						Ok(AstNode::FunctionCall {
							name: name.clone(),
							arguments: transformed_arguments,
						})
					},
					
					AstNode::BinaryOperation {
						operation,
						left,
						right,
					} => {
						let transformed_left = Self::transform_recursive(*left.clone())?;
						let transformed_right = Self::transform_recursive(*right.clone())?;
						
						match operation {
							// Rule 1: A - B  =>  A + (-B)
							BinaryOperationKind::Minus => {
								match &transformed_right {
									AstNode::Number(number) if number.is_sign_negative() => {
										// Rule 1: A - (-B)  =>  A + B
										Ok(AstNode::BinaryOperation {
											operation: BinaryOperationKind::Plus,
											left: Box::new(transformed_left),
											right: Box::new(AstNode::Number(f64::abs(*number))),
										})
									},
									AstNode::Number(number) => {
										// Rule 1: A - B  =>  A + (-B)
										Ok(AstNode::BinaryOperation {
											operation: BinaryOperationKind::Plus,
											left: Box::new(transformed_left),
											right: Box::new(AstNode::Number(-number)),
										})
									},
									_ => {
										let new_right = AstNode::UnaryOperation {
											operation: UnaryOperationKind::Minus,
											expression: Box::new(transformed_right),
										};
										
										let result_node = AstNode::BinaryOperation {
											operation: BinaryOperationKind::Plus,
											left: Box::new(transformed_left),
											right: Box::new(new_right),
										};
										Self::transform_recursive(result_node)
									},
								}
							},
							
							// Rule 2: A / B  =>  A * (1 / B)
							BinaryOperationKind::Divide => {
								if let AstNode::Number(number) = transformed_left
								&& number == 1.0
								{
									return Ok(AstNode::BinaryOperation {
										operation: BinaryOperationKind::Divide,
										left: Box::new(transformed_left),
										right: Box::new(transformed_right),
									});
								}
								
								if let AstNode::Number(number) = transformed_right {
									return if number == 0.0 {
										Err(AstError::DivisionByZero(node))
									} else {
										Ok(AstNode::BinaryOperation {
											operation: BinaryOperationKind::Multiply,
											left: Box::new(transformed_left),
											right: Box::new(AstNode::Number(1.0 / number)),
										})
									};
								}
								Ok(AstNode::BinaryOperation {
									operation: BinaryOperationKind::Multiply,
									left: Box::new(transformed_left),
									right: Box::new(AstNode::BinaryOperation {
										operation: BinaryOperationKind::Divide,
										left: Box::new(AstNode::Number(1.0)), // "1"
										right: Box::new(transformed_right),   // "B"
									}),
								})
							},
							
							// Other operations (Plus, Multiply, And, Or)
							// left without editing, but with transformed kids.
							_ => Ok(AstNode::BinaryOperation {
								operation: operation.clone(),
								left: Box::new(transformed_left),
								right: Box::new(transformed_right),
							}),
						}
					},
					
					AstNode::ArrayAccess {
						identifier,
						indices,
					} => {
						let mut transformed_indices = vec![];
						for index in indices {
							transformed_indices.push(Self::transform_recursive(index.clone())?);
						}
						Ok(AstNode::ArrayAccess {
							identifier: identifier.clone(),
							indices: transformed_indices,
						})
					},
				}
			}
		}
		
		pub fn report_success(tree: &AbstractSyntaxTree) {
			log::warn!(
			"{} {}.",
			"Transformed Abstract-Syntax Tree generation",
			"success".bold().green()
			);
			log::info!("{}", tree.pretty_print());
		}
		
		pub fn report_error(error: AstError) {
			log::error!(
			"{} {}",
			"Transformed Abstract-Syntax Tree generation:".bold().red(),
			error
			);
		}
	\end{lstlisting}
	
	\textbf{balancer.rs}
	
	\begin{lstlisting}[language=Rust]
		use crate::compiler::ast::tree::{
			AbstractSyntaxTree, AstError, AstNode, BinaryOperationKind,
		};
		use colored::Colorize;
		use std::collections::VecDeque;
		
		impl AbstractSyntaxTree {
			pub fn balance(self) -> Result<Self, AstError> {
				let peek = Self::balance_tree(self.peek)?;
				
				Ok(Self::from_node(peek))
			}
			
			pub fn balance_tree(node: AstNode) -> Result<AstNode, AstError> {
				match node {
					// Base cases, already balanced.
					AstNode::Number(_) | AstNode::Identifier(_) | AstNode::StringLiteral(_) => {
						Ok(node)
					},
					
					// Recursive cases for other node types.
					AstNode::UnaryOperation {
						operation,
						expression,
					} => Ok(AstNode::UnaryOperation {
						operation,
						expression: Box::new(Self::balance_tree(*expression)?),
					}),
					
					AstNode::FunctionCall { name, arguments } => {
						let mut balanced_arguments: Vec<AstNode> = vec![];
						for arg in arguments {
							balanced_arguments.push(Self::balance_tree(arg)?);
						}
						
						Ok(AstNode::FunctionCall {
							name,
							arguments: balanced_arguments,
						})
					},
					
					AstNode::ArrayAccess {
						identifier,
						indices,
					} => {
						let mut balanced_indices: Vec<AstNode> = vec![];
						for index in indices {
							balanced_indices.push(Self::balance_tree(index)?);
						}
						
						Ok(AstNode::ArrayAccess {
							identifier,
							indices: balanced_indices,
						})
					},
					
					// Main logic: Binary operations
					AstNode::BinaryOperation {
						operation,
						left,
						right,
					} => {
						match operation {
							BinaryOperationKind::Plus | BinaryOperationKind::Multiply => {
								let mut operands = Vec::new();
								Self::collect_operands(
								AstNode::BinaryOperation {
									operation: operation.clone(),
									left,
									right,
								},
								operation.clone(),
								&mut operands,
								);
								
								let mut balanced_operands = Vec::new();
								for operand in operands {
									balanced_operands.push(Self::balance_tree(operand)?);
								}
								
								Self::build_balanced_tree(balanced_operands, operation)
							},
							
							// Other operations (And, Or, etc.) are not associative
							// in the arithmetic context. Just return them
							// with already balanced children.
							_ => {
								let balanced_left = Self::balance_tree(*left)?;
								let balanced_right = Self::balance_tree(*right)?;
								Ok(AstNode::BinaryOperation {
									operation,
									left: Box::new(balanced_left),
									right: Box::new(balanced_right),
								})
							},
						}
					},
				}
			}
			
			/// Making flatten tree.
			/// Recursively "unfolds" a chain of associative operations
			/// into a flat list. For example, the tree `(a + (b + c)) + d`
			/// with `op_kind = Plus` will be "flattened" into the list `[a, b, c, d]`.
			fn collect_operands(
			node: AstNode, op_kind: BinaryOperationKind, operands: &mut Vec<AstNode>,
			) {
				match node {
					// If operation node is the same as we are looking for...
					AstNode::BinaryOperation {
						operation,
						left,
						right,
					} if operation == op_kind => {
						// ... we recursively collect operands from both sides.
						Self::collect_operands(*left, op_kind.clone(), operands);
						Self::collect_operands(*right, op_kind.clone(), operands);
					},
					// If operation node is different, or it's a leaf node...
					// or just operand (a, b, c...),
					// then it is a "leaf" for *this* chain.
					// We add it to the list.
					_ => {
						operands.push(node);
					},
				}
			}
			
			/// Building balanced tree
			/// Taking a flat list of operands and constructing
			/// a binary tree of minimal height using a queue-based algorithm.
			/// For example, `[a, b, c, d, e]` becomes `((a + b) + (c + d)) + e`
			/// (or a similar balanced structure).
			fn build_balanced_tree(
			operands: Vec<AstNode>, op_kind: BinaryOperationKind,
			) -> Result<AstNode, AstError> {
				if operands.is_empty() {
					return Err(AstError::CannotBuildEmptyTree);
				}
				
				// Making a queue from the list of operands
				let mut queue: VecDeque<AstNode> = operands.into();
				
				// While more than one node remains in the queue...
				while queue.len() > 1 {
					let level_size = queue.len();
					
					// Process the current level of the tree:
					for _ in 0..(level_size / 2) {
						// Take two nodes from the front of the queue...
						let left = queue.pop_front().ok_or(AstError::FailedPopFromQueue)?;
						let right = queue.pop_front().ok_or(AstError::FailedPopFromQueue)?;
						
						// ...create a new binary operation node combining them...
						let new_node = AstNode::BinaryOperation {
							operation: op_kind.clone(),
							left: Box::new(left),
							right: Box::new(right),
						};
						
						// .. and put the new node at the back of the queue
						// (it will be an operand for the next, higher level)
						queue.push_back(new_node);
					}
					
					// if level_size is odd...
					if !level_size.is_multiple_of(2) {
						// ...one node remains at the front of the queue.
						// We simply move it to the back,
						// so it can participate in the next iteration (next level).
						let odd_one_out =
						queue.pop_front().ok_or(AstError::FailedPopFromQueue)?;
						queue.push_back(odd_one_out);
					}
				}
				
				// When only one node remains in the queue,
				// it is the root of the balanced tree.
				queue.pop_front().ok_or(AstError::FailedPopFromQueue)
			}
		}
		
		pub fn report_success(tree: &AbstractSyntaxTree) {
			log::warn!(
			"{} {}.",
			"Balanced Abstract-Syntax Tree generation",
			"success".bold().green()
			);
			log::info!("{}", tree.pretty_print());
		}
		
		pub fn report_error(error: AstError) {
			log::error!("{} {}", "Balanced AST error:".bold().red(), error);
		}
	\end{lstlisting}
\end{document}