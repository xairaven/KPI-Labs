\documentclass[14pt, a4paper]{extreport}

% Constant
\newcommand{\CourseTitle}{Програмування комп'ютерних та віртуальних мереж}
\newcommand{\Variant}{5}
\newcommand{\StudentGroup}{ІМ-51мн}
\newcommand{\CourseNumber}{1}
\newcommand{\StudentName}{Ковальов Олександр}
\newcommand{\Teacher}{доцент, Долголенко Олександр Миколайович}
\newcommand{\Year}{2025}
% Variable
\newcommand{\LabNumber}{3}
\newcommand{\Topic}{Створення лінійної SDN мережі на базі OpenFlow}
\newcommand{\SubmissionDate}{16.10.2025}

\usepackage{caption}
\usepackage{enumitem}
\usepackage{fancyhdr}
\usepackage{float}
\usepackage{fontspec}
\usepackage{geometry}
\usepackage{graphicx}
\usepackage{hyperref}
\usepackage{hyphenat}
\usepackage{indentfirst}
\usepackage{listings}
\usepackage{polyglossia}
\usepackage{setspace}
\usepackage{unicode-math}
\usepackage{xcolor}



\defaultfontfeatures{Ligatures=TeX}
\graphicspath{{./images/}}
\fancyhf{}
\hypersetup{unicode=true, colorlinks=true, linkcolor=black, urlcolor=black}
\pagestyle{fancy}
\setlength{\headheight}{18pt}
\setmainfont{Times New Roman}
\setmathfont{Latin Modern Math}
\setmonofont[Scale=0.8]{DejaVu Sans Mono}
\setotherlanguage{english}
\setsansfont{Arial}
\onehalfspacing

\renewcommand\headrulewidth{0pt}
\cfoot{\thepage}

\geometry{
	a4paper,
	left=25mm,
	right=10mm,
	top=15mm,
	bottom=15mm,
}

\definecolor{codegreen}{rgb}{0,0.6,0}
\definecolor{codegray}{rgb}{0.5,0.5,0.5}
\definecolor{codepurple}{rgb}{0.58,0,0.82}
\definecolor{backcolour}{rgb}{ 0.976, 0.976, 0.976 }

\lstdefinestyle{mystyle}{
	backgroundcolor=\color{backcolour},   
	commentstyle=\color{codegreen},
	keywordstyle=\color{magenta},
	numberstyle=\tiny\color{codegray},
	stringstyle=\color{codepurple},
	basicstyle=\ttfamily\footnotesize,
	breakatwhitespace=false,         
	breaklines=true,                 
	captionpos=b,                    
	keepspaces=true,                 
	numbers=left,                    
	numbersep=5pt,                  
	showspaces=false,                
	showstringspaces=false,
	showtabs=false,                  
	tabsize=2
}

\lstset{style=mystyle}

\setlist[enumerate]{
	itemsep=0.0\baselineskip,
	left=1.25cm,
	rightmargin=10mm,
	labelsep=0.5cm,
	listparindent=1.25cm,
	parsep=0pt
}

\hypersetup{
	colorlinks=true,
	linkcolor=blue,
	filecolor=magenta,      
	urlcolor=cyan
}


\begin{document}
	\tolerance=350 % Or just increase the number
	\emergencystretch=3em
	
	\begin{titlepage}
		\begin{center}
			{Національний технічний університет України\\
				«Київський політехнічний інститут імені Ігоря Сікорського» \\[1.0em] }
			{Факультет інформатики та обчислювальної техніки\\}
			{Кафедра обчислювальної техніки \\[5.0em]}
			
			{\textbf{ЗВІТ}\\[1em]}
			{\textbf{з лабораторної роботи №\LabNumber} \\}
			{\textbf{з дисципліни "\CourseTitle"} \\[2.0em]}
			
			{\textbf{Тема: \Topic} \\[2.0em]}
			
			{\textbf{Варіант №\Variant} \\[5.0em]}
			
			\begin{flushright}
				Виконав: \\
				Студент \CourseNumber{} курсу, групи \StudentGroup \\
				\StudentName \\[2.0em]
			\end{flushright}
			
			\begin{flushright}
				Перевірив: \\
				\Teacher \\[2.0em]
			\end{flushright}
			
			\begin{flushright}
				Дата здачі: \SubmissionDate \\[5.0em]
			\end{flushright}
		
			\vfill
			КИЇВ -- \Year
		\end{center}
	\end{titlepage}
	
	\setlength{\parindent}{1.25cm}
	
	\textbf{Мета роботи.} Налаштувати та дослідити деревовидну топологію SDN мережі з різними рівнями комутаторів і підключених хостів, перевірити її працездатність і проаналізувати трафік за допомогою Wireshark.
	
	\textbf{Завдання:} За допомогою скрипту \texttt{miniedit.py} Mininet (GUI) створити деревовидну топологію SDN мережі, що має глибину ієрархії комутаторів \texttt{depth = i \% 3 + 2}, а число підключених до кожного з них комутаторів, або хостів \texttt{fanout = i \% 2 + 2}, де \texttt{i} -- номер в списку групи, а хости підключені тільки до комутаторів нижнього рівня. Продемонструвати працездатність топології з використанням Wireshark.
	
	\begin{center}
		\textbf{Хід роботи.}
	\end{center}
	
	Номер в списку групи -- 5, тому, відповідно, глибина ієрархії комутаторів = 4, а число пристроїв, підключених до кожного з них = 3.
	
	Для того, щоб запустити Miniedit, треба налаштувати Х11-форвардинг та мати встановлену змінну \texttt{DISPLAY}. Якщо все налаштовано правильно -- буде відображений інтерфейс:
	
	\begin{figure}[H]
		\centering
		\includegraphics[height=12cm]{01} 
	\end{figure}
	
	\pagebreak
	
	Була створена топологія з 15 комутаторами та 24 хостами:
	
	\begin{figure}[H]
		\centering
		\includegraphics[height=5cm]{02} 
	\end{figure}
	
	Цей приклад був створений вручну. Якщо слідувати завданню, то у кожного комутатора повинно бути 3 "спадкоємці" -- загалом, виходить що топологія повинна мати 40 комутаторів та 81 хост. Це доволі складно реалізувати, тому був написаний скрипт, який реалізує JSON файл зі заданою топологією згідно завдання. Далі, його можна завантажити і працювати безпосередньо в середовищі MiniEdit.
	
	Приклад застосування показаний на скріншоті. Аргумент "\texttt{--branching}" відповідає за кількість дочірніх комутаторів, "\texttt{--levels}" визначає кількість рівнів ієрархії, а "\texttt{--hosts}" -- по скільки хостів треба мати кожному комутатору на нижньому рівні.
	
	\begin{figure}[H]
		\centering
		\includegraphics[height=2cm]{03} 
	\end{figure}
	
	Після цього, власне, можна спробувати запустити топологію в графічному інтерфейсі. Відразу стає зрозуміло, що виставляти вручну всі пристрої було б недоцільною задачею.
	
	\begin{figure}[H]
		\centering
		\includegraphics[height=7cm]{04} 
	\end{figure}
	
	Ще одною проблемою є застарілість програмного забезпечення системи Mininet. Проект вже давно закинутий, а деякі функції, такі як, наприклад, збереження топології в файл -- не працюють. Тому, автором лабораторної роботи було створене рішення -- декілька рядків програми Miniedit були змінені, а самі зміни були надіслані авторам проекту. Їх можна знайти за \href{https://github.com/mininet/mininet/pull/1245}{цим посиланням}. Також, доцільно весь файл Miniedit на віртуальній машині замінити на новий, безпосередньо з репозиторію Github. Якщо ж при цьому виникне помилка пов'язана з файлом \texttt{utils.py} -- то його теж бажано замінити новим кодом, за \href{https://github.com/mininet/mininet/blob/master/mininet/util.py}{даним посиланням}.
	
	Щоб запустити симуляцію, потрібно натиснути "Run". Щоб переглянути всі події в консолі, треба увімкнути її перед цим. Ця опція доступна в меню "Edit" --> "Preferences" --> "Start CLI":
	
	\begin{figure}[H]
		\centering
		\includegraphics[height=8cm]{05} 
	\end{figure}
	
	Щоб мати змогу протестувати мережу, треба перейти в термінал:
	
	\begin{figure}[H]
		\centering
		\includegraphics[height=5cm]{06} 
	\end{figure}
	
	\pagebreak
	
	Далі, повертаючись до терміналу, можна почати тестувати мережу. Наприклад, пропінгувати хости з різних кінців дерева:
	
	\begin{figure}[H]
		\centering
		\includegraphics[height=10cm]{07} 
	\end{figure}
	
	В окремому підключенні до віртуальної машини запустимо Wireshark:
	
	\begin{lstlisting}[language=Bash]
		mininet@mininet-vm:~$ sudo -E wireshark &
		[1] 2958\end{lstlisting}
		
	Всі пакети відображаються -- мережа працює. 
	
	\begin{figure}[H]
		\centering
		\includegraphics[height=10cm]{08} 
	\end{figure}
	
	Окрім цього, можна помітити цікавий момент -- дійсно, всі пакети дублюються і слугують корисним навантаженням на рівні OpenFlow.
	
	\begin{figure}[H]
		\centering
		\includegraphics[height=16cm]{09} 
	\end{figure}
	
	\textbf{Висновок.}
	У роботі створено та протестовано деревоподібну SDN-топологію -- 40 комутаторів і 81 хост. Тестування зв’язності (\texttt{pingall}, пінги між листовими вузлами) підтвердило коректну роботу мережі; у Wireshark зафіксовано OpenFlow-повідомлення (\texttt{PACKET\_IN}, \texttt{PACKET\_OUT}, \texttt{FLOW\_MOD}), що свідчить про роботу контролера та динамічне встановлення правил. Завдання виконано.
		
	\pagebreak
	
	\begin{center}
		\textbf{Лістинг.}
	\end{center}
	
	\textbf{gui\_topology\_creator.py}
	
	\begin{lstlisting}[language=Python]
		#!/usr/bin/env python3
		"""
		gui_topology_creator.py
		
		Creating tree topology and saving it as JSON (.mn) file.
		
		Usage example:
		python gui_topology_creator.py --branching 2 --levels 3 --hosts 2 --out topology.mn
		"""
		
		import json
		import argparse
		import math
		
		BASE_APPLICATION = {
			"dpctl": "",
			"ipBase": "10.0.0.0/8",
			"netflow": {
				"nflowAddId": "0",
				"nflowTarget": "",
				"nflowTimeout": "600"
			},
			"openFlowVersions": {
				"ovsOf10": "1",
				"ovsOf11": "0",
				"ovsOf12": "0",
				"ovsOf13": "0"
			},
			"sflow": {
				"sflowHeader": "128",
				"sflowPolling": "30",
				"sflowSampling": "400",
				"sflowTarget": ""
			},
			"startCLI": "0",
			"switchType": "ovs",
			"terminalType": "xterm"
		}
		
		CONTROLLER_TEMPLATE = {
			"opts": {
				"controllerProtocol": "tcp",
				"controllerType": "ref",
				"hostname": "c0",
				"remoteIP": "127.0.0.1",
				"remotePort": 6633
			},
			"x": "0.0",
			"y": "0.0"
		}
		
		
		def build_tree(branching: int, levels: int, hosts_per_bottom: int):
			if levels < 1:
				raise ValueError("levels must be >= 1")
			if branching < 1:
				raise ValueError("branching must be >= 1")
			if hosts_per_bottom < 0:
				raise ValueError("hosts_per_bottom must be >= 0")
		
			controllers = [CONTROLLER_TEMPLATE.copy()]
		
			switches = []
			hosts = []
			links = []
		
			# Build tree structure and remember parent->children mapping
			level_nodes = []  # list of lists: level_nodes[0] = [root], ..., level_nodes[-1] = bottom switches
			parent_children = {}  # parent_name -> list of child switch names
			next_switch_id = 1
		
			# create root switch
			root_name = f"s{next_switch_id}"
			level_nodes.append([root_name])
			switch_by_name = {}
			switch_obj = {
				"number": str(next_switch_id),
				"opts": {
					"controllers": ["c0"],
					"hostname": root_name,
					"nodeNum": next_switch_id,
					"switchType": "default"
				},
				"x": "0.0",
				"y": "0.0"
			}
			switches.append(switch_obj)
			switch_by_name[root_name] = switch_obj
			next_switch_id += 1
			
			# create child switch levels and links (child -> parent)
			for lvl in range(1, levels):
				parents = level_nodes[lvl - 1]
				this_level = []
				for p in parents:
					for i in range(branching):
						name = f"s{next_switch_id}"
						this_level.append(name)
						obj = {
							"number": str(next_switch_id),
							"opts": {
								"controllers": ["c0"],
								"hostname": name,
								"nodeNum": next_switch_id,
								"switchType": "default"
							},
							"x": "0.0",
							"y": "0.0"
						}
						switches.append(obj)
						switch_by_name[name] = obj
						# record parent->child
						parent_children.setdefault(p, []).append(name)
						# link child -> parent
						links.append({
							"dest": p,
							"opts": {},
							"src": name
						})
						next_switch_id += 1
				level_nodes.append(this_level)
			
			# attach hosts to every switch in the last level (bottom)
			host_id = 1
			bottom_level = level_nodes[-1]
			for s in bottom_level:
				for i in range(hosts_per_bottom):
					hname = f"h{host_id}"
					hosts.append({
						"number": str(host_id),
						"opts": {
							"hostname": hname,
							"nodeNum": host_id,
							"sched": "host"
						},
						"x": "0.0",  # will set numeric x immediately
						"y": "0.0"   # will set actual y after switch y computed
					})
					# link host -> its switch (src host, dest switch)
					links.append({
						"dest": s,
						"opts": {},
						"src": hname
					})
					host_id += 1
			
			# Coordinate policy parameters
			host_x_start = 10.0
			host_x_step = 15.0    # produces 10,20,30,...
			controller_y = 20.0
			switch_y_start = 80.0  # root level y
			y_spacing = 100.0      # distance between switch levels
			host_y_offset = 80.0   # hosts below the bottom switches
			
			# assign X to hosts in a simple left-to-right sequence: 10,20,30,...
			total_hosts = len(hosts)
			host_xs = [host_x_start + i * host_x_step for i in range(total_hosts)]
			for idx, h in enumerate(hosts):
				h["x"] = f"{host_xs[idx]:.1f}"
				# y left as placeholder; we'll compute after computing switch Y
			
			# compute X for bottom switches:
			switch_xs = {}  # name -> numeric x
			# hosts are ordered in the same sequence we created them: for each bottom switch it has hosts_per_bottom hosts sequentially
			if hosts_per_bottom > 0:
				# mapping: for each bottom switch in order, take next hosts_per_bottom host_xs
				hi = 0
				for s in bottom_level:
					slice_xs = host_xs[hi:hi + hosts_per_bottom]
					if slice_xs:
						sx = sum(slice_xs) / len(slice_xs)
					else:
						sx = host_x_start + hi * host_x_step
					switch_xs[s] = sx
					hi += hosts_per_bottom
			else:
				# no hosts attached: spread bottom switches evenly across a small span
				n = len(bottom_level)
				if n == 1:
					switch_xs[bottom_level[0]] = host_x_start
				else:
					span = (n - 1) * host_x_step * 2
					for i, s in enumerate(bottom_level):
						switch_xs[s] = host_x_start - span / 2 + i * (span / max(1, n - 1))
			
			# compute X for higher-level switches by averaging their children's X
			# process levels from bottom-1 up to root
			for lvl in range(levels - 2, -1, -1):
				for parent in level_nodes[lvl]:
					children = parent_children.get(parent, [])
					if not children:
						# isolated parent (no children) - keep its X if exists or set to 0
						sx = switch_xs.get(parent, host_x_start)
					else:
						child_xs = [switch_xs[child] for child in children]
						sx = sum(child_xs) / len(child_xs)
					switch_xs[parent] = sx
			
			# now set switch objects' x and y strings
			for lvl, nodes in enumerate(level_nodes):
				y = switch_y_start + lvl * y_spacing
				for name in nodes:
					sx = switch_xs.get(name, host_x_start)
					sw = switch_by_name[name]
					sw["x"] = f"{sx:.1f}"
					sw["y"] = f"{y:.1f}"
			
			# set hosts' y based on their parent switch Y (each host was sequentially assigned to bottom switches)
			if hosts_per_bottom > 0:
				hi = 0
				for s in bottom_level:
					parent_sw = switch_by_name[s]
					py = float(parent_sw["y"])
					for i in range(hosts_per_bottom):
					if hi >= len(hosts):
						break
					hosts[hi]["y"] = f"{py + host_y_offset:.1f}"
					hi += 1
			else:
				# no hosts, nothing to update
				pass
			
			# controller coordinates: center above root
			root_x = switch_xs.get(level_nodes[0][0], host_x_start)
			controllers[0]["x"] = f"{root_x:.1f}"
			controllers[0]["y"] = f"{controller_y:.1f}"
			
			result = {
				"application": BASE_APPLICATION,
				"controllers": controllers,
				"hosts": hosts,
				"links": links,
				"switches": switches,
				"version": "2"
			}
			return result
			
			
			def main():
				p = argparse.ArgumentParser(description="Build tree topology JSON")
				p.add_argument("--branching", "-b", type=int, default=2,
				help="number of child switches per switch (branching factor)")
				p.add_argument("--levels", "-l", type=int, default=2,
				help="number of switch levels (1 = root only)")
				p.add_argument("--hosts", "-s", dest="hosts", type=int, default=1,
				help="number of hosts per bottom-level switch")
				p.add_argument("--out", "-o", type=str, default="topology.json",
				help="output JSON filename")
				args = p.parse_args()
			
			topo = build_tree(args.branching, args.levels, args.hosts)
			
			with open(args.out, "w", encoding="utf-8") as f:
			json.dump(topo, f, ensure_ascii=False, indent=4)
			print(f"Saved topology to {args.out} (branching={args.branching}, levels={args.levels}, hosts_per_bottom={args.hosts})")
		
		
		if __name__ == "__main__":
			main()
		\end{lstlisting}
\end{document}