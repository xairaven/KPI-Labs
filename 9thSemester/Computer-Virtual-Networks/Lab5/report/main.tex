\documentclass[14pt, a4paper]{extreport}

% Constant
\newcommand{\CourseTitle}{Програмування комп'ютерних та віртуальних мереж}
\newcommand{\Variant}{5}
\newcommand{\StudentGroup}{ІМ-51мн}
\newcommand{\CourseNumber}{1}
\newcommand{\StudentName}{Ковальов Олександр}
\newcommand{\Teacher}{доцент, Долголенко Олександр Миколайович}
\newcommand{\Year}{2025}
% Variable
\newcommand{\LabNumber}{3}
\newcommand{\Topic}{Створення лінійної SDN мережі на базі OpenFlow}
\newcommand{\SubmissionDate}{16.10.2025}

\usepackage{caption}
\usepackage{enumitem}
\usepackage{fancyhdr}
\usepackage{float}
\usepackage{fontspec}
\usepackage{geometry}
\usepackage{graphicx}
\usepackage{hyperref}
\usepackage{hyphenat}
\usepackage{indentfirst}
\usepackage{listings}
\usepackage{polyglossia}
\usepackage{setspace}
\usepackage{unicode-math}
\usepackage{xcolor}



\defaultfontfeatures{Ligatures=TeX}
\graphicspath{{./images/}}
\fancyhf{}
\hypersetup{unicode=true, colorlinks=true, linkcolor=black, urlcolor=black}
\pagestyle{fancy}
\setlength{\headheight}{18pt}
\setmainfont{Times New Roman}
\setmathfont{Latin Modern Math}
\setmonofont[Scale=0.8]{DejaVu Sans Mono}
\setotherlanguage{english}
\setsansfont{Arial}
\onehalfspacing

\renewcommand\headrulewidth{0pt}
\cfoot{\thepage}

\geometry{
	a4paper,
	left=25mm,
	right=10mm,
	top=15mm,
	bottom=15mm,
}

\definecolor{codegreen}{rgb}{0,0.6,0}
\definecolor{codegray}{rgb}{0.5,0.5,0.5}
\definecolor{codepurple}{rgb}{0.58,0,0.82}
\definecolor{backcolour}{rgb}{ 0.976, 0.976, 0.976 }

\lstdefinestyle{mystyle}{
	backgroundcolor=\color{backcolour},   
	commentstyle=\color{codegreen},
	keywordstyle=\color{magenta},
	numberstyle=\tiny\color{codegray},
	stringstyle=\color{codepurple},
	basicstyle=\ttfamily\footnotesize,
	breakatwhitespace=false,         
	breaklines=true,                 
	captionpos=b,                    
	keepspaces=true,                 
	numbers=left,                    
	numbersep=5pt,                  
	showspaces=false,                
	showstringspaces=false,
	showtabs=false,                  
	tabsize=2
}

\lstset{style=mystyle}

\setlist[enumerate]{
	itemsep=0.0\baselineskip,
	left=1.25cm,
	rightmargin=10mm,
	labelsep=0.5cm,
	listparindent=1.25cm,
	parsep=0pt
}

\hypersetup{
	colorlinks=true,
	linkcolor=blue,
	filecolor=magenta,      
	urlcolor=cyan
}


\begin{document}
	\tolerance=350 % Or just increase the number
	\emergencystretch=3em
	
	\begin{titlepage}
		\begin{center}
			{Національний технічний університет України\\
				«Київський політехнічний інститут імені Ігоря Сікорського» \\[1.0em] }
			{Факультет інформатики та обчислювальної техніки\\}
			{Кафедра обчислювальної техніки \\[5.0em]}
			
			{\textbf{ЗВІТ}\\[1em]}
			{\textbf{з лабораторної роботи №\LabNumber} \\}
			{\textbf{з дисципліни "\CourseTitle"} \\[2.0em]}
			
			{\textbf{Тема: \Topic} \\[2.0em]}
			
			{\textbf{Варіант №\Variant} \\[5.0em]}
			
			\begin{flushright}
				Виконав: \\
				Студент \CourseNumber{} курсу, групи \StudentGroup \\
				\StudentName \\[2.0em]
			\end{flushright}
			
			\begin{flushright}
				Перевірив: \\
				\Teacher \\[2.0em]
			\end{flushright}
			
			\begin{flushright}
				Дата здачі: \SubmissionDate \\[5.0em]
			\end{flushright}
		
			\vfill
			КИЇВ -- \Year
		\end{center}
	\end{titlepage}
	
	\setlength{\parindent}{1.25cm}
	
	\textbf{Мета роботи.} Практична реалізація логіки (на Python/SDN контролері) для забезпечення заданої продуктивності та оптимізації трафіку в каналах розробленої топології мережі ЦОД, використовуючи добовий час і день тижня як керуючі параметри.
	
	\textbf{Завдання:} Керування продуктивністю каналів передачі повідомлень в залежності від добового часу та дня тижня у SDN мережі центру обробки даних (Data Center). Підключити зовнішній контролер, створити скрипт \texttt{*.py}, що реалізує логіку продуктивності каналів передачі повідомлень в залежності від добового часу та дня тижня у SDN мережі Data Center з розробленою топологією.
	
	\begin{center}
		\textbf{Хід роботи.}
	\end{center}
	
	Для виконання поставленого завдання було розроблено програмну емуляцію мережі центру обробки даних (Data Center) з використанням технології програмно-конфігурованих мереж (SDN). Теоретичною основою для побудови топології слугували матеріали статті "A Scalable, Commodity Data Center Network Architecture" авторів Al-Fares, Loukissas та Vahdat. У цій праці описується архітектура Fat-Tree, яка дозволяє ефективно масштабувати пропускну здатність кластерів, використовуючи стандартні комутатори. Враховуючи особливості середовища емуляції Mininet та необхідність забезпечення стабільної маршрутизації без використання складних протоколів запобігання петель (на кшталт STP) у рамках лабораторної роботи, було вирішено реалізувати ієрархічну деревовидну структуру (Tree Topology), тобто модифікацію Fat-Tree, яка зберігає логіку рівнів Core, Aggregation та Edge, описану в джерелах, але гарантує відсутність фізичних циклів і штормів широкомовного трафіку.
	
	Реалізація топології була виконана мовою Python у скрипті \texttt{dc\_topo.py}. У коді було створено клас, що успадковує функціонал базового класу \texttt{Topo} бібліотеки \texttt{Mininet}. Структура мережі складається з трьох рівнів комутаторів. На верхньому рівні розташовано один кореневий комутатор (Core Switch), який забезпечує зв'язок між підмережами. До нього підключено чотири комутатори рівня агрегації (Aggregation Switches). Кожен комутатор агрегації, своєю чергою, з'єднаний з двома комутаторами рівня доступу (Edge Switches), до яких безпосередньо підключаються кінцеві хости. Загалом у топології задіяно 13 комутаторів та 16 хостів. Важливим аспектом реалізації стало використання методу \texttt{addSwitch} із зазначенням протоколу OpenFlow 1.3, оскільки застарілі версії протоколу не підтримують необхідний для виконання завдання функціонал \texttt{Meters} (лічильників трафіку). Для забезпечення миттєвої зв'язності між хостами та уникнення помилок ARP resolution (проблема "Destination Host Unreachable") у скрипті запуску було активовано функцію \texttt{net.staticArp()}, яка автоматично заповнює ARP-таблиці хостів, емулюючи роботу служби каталогів у реальному ЦОД.
	
	\begin{figure}[H]
		\centering
		\includegraphics[height=17cm]{01} 
	\end{figure}
	
	Наступним етапом стала розробка логіки керування продуктивністю каналів. Для цього було створено скрипт зовнішнього контролера \texttt{time\_qos\_controller.py} на базі фреймворку Ryu. Головна мета контролера полягала в динамічній зміні пропускної здатності каналів залежно від поточного часу та дня тижня. Алгоритм роботи контролера базується на використанні окремого потоку виконання, який з інтервалом у кілька секунд перевіряє системний час. Логіка перевірки визначає, чи є поточний момент "робочим часом" (понеділок - п'ятниця, з 09:00 до 18:00). Якщо умова виконується, мережа переходить у режим високої продуктивності, інакше -- у режим обмеженої швидкості.
	
	Для технічної реалізації обмеження швидкості, замість повного відкидання пакетів, було використано механізм OpenFlow Meters. У коді контролера визначено функцію, яка надсилає повідомлення \texttt{OFPMeterMod} на комутатори. Ця функція налаштовує лічильник (\texttt{Meter}) з певним ідентифікатором та типом смуги \texttt{OFPMeterBandDrop}. Це означає, що пакети, частота надходження яких перевищує встановлений поріг (\texttt{rate}), будуть відкидатися комутатором, що для кінцевого користувача виглядає як зниження швидкості передачі даних, а не розрив з'єднання. У скрипті задано дві константи швидкості: висока для робочих годин (100 Мбіт/с) та низька для неробочого часу (1 Мбіт/с).
	
	Тестування механізму керування швидкістю за допомогою утиліти \texttt{iperf}. Оскільки тестування проводилося у вечірній час (поза межами робочого інтервалу), то час було змінено вручну за допомогою зміну часових зон \texttt{timedatectl}. У вечірні години контролер автоматично перейшов у режим обмеження. Під час вимірювання пропускної здатності між хостами \texttt{h1} та \texttt{h16}, які знаходяться у різних підмережах, утиліта \texttt{iperf} зафіксувала швидкість передачі даних на рівні близько 1 Мбіт/с. Це підтверджує, що механізм QoS спрацював коректно: швидкість була "обрізана" до заданого ліміту, але з'єднання не було розірвано. Логи контролера також відображали зміну контексту часу та відправку команд на оновлення параметрів лічильників на комутаторах. Те ж саме відбулося в "робочі" години, але зі швидкістю 130 Мбіт/с.
	
	Тестування в робочі години:
	
	\begin{figure}[H]
		\centering
		\includegraphics[height=10cm]{02} 
	\end{figure}
	
	В неробочі:
	
	\begin{figure}[H]
		\centering
		\includegraphics[height=10cm]{03} 
	\end{figure}
	
	Окрім керування QoS, контролер виконує функцію L2-комутації (Learning Switch). Обробник події \texttt{EventOFPPacketIn} аналізує вхідні пакети, запам'ятовує MAC-адреси відправників та відповідні порти, заповнюючи таблицю комутації. При встановленні потоків (Flow Entries) для передачі даних контролер додає інструкцію \texttt{OFPInstructionMeter}, яка прив'язує трафік до раніше створеного лічильника. Таким чином, будь-який трафік між хостами проходить через механізм обмеження швидкості.
	
	Перевірка працездатності системи проводилася у кілька етапів. Спочатку було запущено контролер Ryu, а потім -- скрипт топології Mininet. Після ініціалізації мережі було виконано команду \texttt{pingall} у консолі Mininet. Результати показали 100\% успішність проходження пакетів, що підтвердило коректність побудови топології, відсутність петель та правильну роботу механізму Static ARP.
	
	\begin{figure}[H]
		\centering
		\includegraphics[height=5cm]{04} 
	\end{figure}
	
	\textbf{Висновок.}
	У ході виконання лабораторної роботи було успішно спроектовано та реалізовано програмно-конфігуровану мережу центру обробки даних. Спираючись на теоретичні засади архітектури Data Center Networks, зокрема концепцію ієрархічної побудови мережі, було створено топологію, що забезпечує повну зв'язність хостів. Використання зовнішнього контролера на базі Ryu дозволило реалізувати інтелектуальне керування трафіком.
	
	Основне завдання роботи -- керування продуктивністю каналів залежно від часу доби -- було виконано за допомогою протоколу OpenFlow 1.3 та механізму Meters. Це дозволило динамічно змінювати параметри пропускної здатності мережі без необхідності перезавантаження обладнання чи розриву активних з'єднань. Експериментальна перевірка підтвердила, що у визначені "неробочі" години швидкість передачі даних обмежується до заданого мінімуму, тоді як у "робочий" час надається повна смуга пропускання. Такий підхід дозволяє гнучко керувати ресурсами мережі ЦОД, адаптуючись до графіку навантаження.
	
	\pagebreak
	
	\begin{center}
		\textbf{Лістинг.}
	\end{center}
	
	\textbf{dc\_topo.py}
	\begin{lstlisting}[language=Python]
		from mininet.topo import Topo
		from mininet.net import Mininet
		from mininet.node import RemoteController, OVSKernelSwitch
		from mininet.cli import CLI
		from mininet.log import setLogLevel, info
		from mininet.link import TCLink

		class SimpleDCTopo(Topo):
			"""
			Simplified Data Center Topology (Tree based).
			No physical loops, so STP is NOT required.
			Structure: Core -> Aggregation -> Edge -> Hosts.
			"""
			def build(self):
				# --- Create Switches ---
				# 1 Core Switch
				c1 = self.addSwitch('c1', dpid='0000000000000001',
									protocols='OpenFlow13')

				# 4 Aggregation Switches
				aggrs = []
				for i in range(4):
					sw = self.addSwitch(f'a{i+1}', dpid=f'000000000000001{i+1}',
										protocols='OpenFlow13')
					aggrs.append(sw)

				# 8 Edge Switches
				edges = []
				for i in range(8):
					sw = self.addSwitch(f'e{i+1}', dpid=f'000000000000002{i+1}',
										protocols='OpenFlow13')
					edges.append(sw)

				# --- Create Links (No Loops) ---

				# 1. Connect Core to all Aggregation switches
				for a_sw in aggrs:
					self.addLink(c1, a_sw)

				# 2. Connect Aggregation to Edge (Simple Tree structure)
				# a1 -> e1, e2; a2 -> e3, e4 ...
				edge_idx = 0
				for a_sw in aggrs:
					self.addLink(a_sw, edges[edge_idx])
					self.addLink(a_sw, edges[edge_idx+1])
					edge_idx += 2

				# 3. Connect Hosts to Edge (2 Hosts per Edge)
				# e1 -> h1, h2; e2 -> h3, h4 ...
				host_count = 1
				for e_sw in edges:
					h1 = self.addHost(f'h{host_count}')
					h2 = self.addHost(f'h{host_count+1}')
					self.addLink(e_sw, h1)
					self.addLink(e_sw, h2)
					host_count += 2

		def run_topology():
			topo = SimpleDCTopo()
			net = Mininet(topo=topo,
						controller=RemoteController,
						switch=OVSKernelSwitch,
						link=TCLink,
						autoSetMacs=True)

			info("**** Starting Network ****\n")
			net.start()

			# Static ARP is helpful to speed up discovery
			info("**** Enabling Static ARP ****\n")
			net.staticArp()

			info("**** Network Ready (No loops, full connectivity) ****\n")
			CLI(net)

			info("**** Stopping Network ****\n")
			net.stop()

		if __name__ == '__main__':
			setLogLevel('info')
			run_topology()


	\end{lstlisting}
	
	\textbf{time\_qos\_controller.py}
	\begin{lstlisting}[language=Python]
		import datetime
		from ryu.base import app_manager
		from ryu.controller import ofp_event
		from ryu.controller.handler import CONFIG_DISPATCHER, MAIN_DISPATCHER
		from ryu.controller.handler import set_ev_cls
		from ryu.ofproto import ofproto_v1_3
		from ryu.lib import hub
		from ryu.lib.packet import packet, ethernet, ether_types

		class TimeBasedQoS(app_manager.RyuApp):
			OFP_VERSIONS = [ofproto_v1_3.OFP_VERSION]

			def __init__(self, *args, **kwargs):
				super(TimeBasedQoS, self).__init__(*args, **kwargs)
				self.mac_to_port = {}
				self.datapaths = {}
				self.meter_id = 1

				# QoS Config (kbps)
				self.BW_HIGH = 100000  # 100 Mbps
				self.BW_LOW = 1000     # 1 Mbps
				self.current_mode = None

				self.monitor_thread = hub.spawn(self._time_monitor)

			@set_ev_cls(ofp_event.EventOFPSwitchFeatures, CONFIG_DISPATCHER)
			def switch_features_handler(self, ev):
				datapath = ev.msg.datapath
				ofproto = datapath.ofproto
				parser = datapath.ofproto_parser

				self.datapaths[datapath.id] = datapath

				# Default flow: Send to Controller
				match = parser.OFPMatch()
				actions = [parser.OFPActionOutput(ofproto.OFPP_CONTROLLER,
												ofproto.OFPCML_NO_BUFFER)]
				self.add_flow(datapath, 0, match, actions, use_meter=False)

				# Init Meter
				is_work_hours = self._check_work_hours()
				rate = self.BW_HIGH if is_work_hours else self.BW_LOW
				self.current_mode = 'work' if is_work_hours else 'off'

				self.logger.info(f"Switch {datapath.id} connected. Init Rate: {rate} kbps")
				self.send_meter_mod(datapath, ofproto.OFPMC_ADD, rate)

			@set_ev_cls(ofp_event.EventOFPPacketIn, MAIN_DISPATCHER)
			def _packet_in_handler(self, ev):
				msg = ev.msg
				datapath = msg.datapath
				ofproto = datapath.ofproto
				parser = datapath.ofproto_parser
				in_port = msg.match['in_port']

				try:
					pkt = packet.Packet(msg.data)
					eth = pkt.get_protocols(ethernet.ethernet)[0]

					if eth.ethertype in [ether_types.ETH_TYPE_LLDP, ether_types.ETH_TYPE_IPV6]:
						return

					dst = eth.dst
					src = eth.src
					dpid = datapath.id

					self.mac_to_port.setdefault(dpid, {})
					self.mac_to_port[dpid][src] = in_port

					if dst in self.mac_to_port[dpid]:
						out_port = self.mac_to_port[dpid][dst]
					else:
						out_port = ofproto.OFPP_FLOOD

					actions = [parser.OFPActionOutput(out_port)]

					# Install Flow with QoS Meter
					if out_port != ofproto.OFPP_FLOOD:
						match = parser.OFPMatch(in_port=in_port, eth_dst=dst, eth_src=src)
						if msg.buffer_id != ofproto.OFP_NO_BUFFER:
							self.add_flow(datapath, 1, match, actions, msg.buffer_id, use_meter=True)
						else:
							self.add_flow(datapath, 1, match, actions, use_meter=True)

					data = None
					if msg.buffer_id == ofproto.OFP_NO_BUFFER:
						data = msg.data
					out = parser.OFPPacketOut(datapath=datapath, buffer_id=msg.buffer_id,
											in_port=in_port, actions=actions, data=data)
					datapath.send_msg(out)
				except:
					return

			def add_flow(self, datapath, priority, match, actions, buffer_id=None, use_meter=True):
				ofproto = datapath.ofproto
				parser = datapath.ofproto_parser
				inst = [parser.OFPInstructionActions(ofproto.OFPIT_APPLY_ACTIONS, actions)]

				if priority > 0 and use_meter:
					inst.append(parser.OFPInstructionMeter(self.meter_id))

				if buffer_id:
					mod = parser.OFPFlowMod(datapath=datapath, buffer_id=buffer_id,
											priority=priority, match=match, instructions=inst)
				else:
					mod = parser.OFPFlowMod(datapath=datapath, priority=priority,
											match=match, instructions=inst)
				datapath.send_msg(mod)

			def send_meter_mod(self, datapath, command, rate_kbps):
				ofproto = datapath.ofproto
				parser = datapath.ofproto_parser
				bands = [parser.OFPMeterBandDrop(rate=rate_kbps, burst_size=10)]
				req = parser.OFPMeterMod(datapath=datapath, command=command,
										flags=ofproto.OFPMF_KBPS, meter_id=self.meter_id, bands=bands)
				datapath.send_msg(req)

			def _check_work_hours(self):
				now = datetime.datetime.now()
				is_weekday = 0 <= now.weekday() <= 4
				is_working_time = 9 <= now.hour < 18

				return is_weekday and is_working_time

			def _time_monitor(self):
				while True:
					is_work_hours = self._check_work_hours()
					new_mode = 'work' if is_work_hours else 'off'

					if new_mode != self.current_mode:
						self.logger.info(f"Time Changed -> {new_mode}. Updating Meters...")
						self.current_mode = new_mode
						new_rate = self.BW_HIGH if is_work_hours else self.BW_LOW
						for dp in self.datapaths.values():
							self.send_meter_mod(dp, dp.ofproto.OFPMC_MODIFY, new_rate)
					hub.sleep(10)
	\end{lstlisting}
\end{document}
