\documentclass[14pt, a4paper]{extreport}

% Constant
\newcommand{\CourseTitle}{Програмування комп'ютерних та віртуальних мереж}
\newcommand{\Variant}{5}
\newcommand{\StudentGroup}{ІМ-51мн}
\newcommand{\CourseNumber}{1}
\newcommand{\StudentName}{Ковальов Олександр}
\newcommand{\Teacher}{доцент, Долголенко Олександр Миколайович}
\newcommand{\Year}{2025}
% Variable
\newcommand{\LabNumber}{3}
\newcommand{\Topic}{Створення лінійної SDN мережі на базі OpenFlow}
\newcommand{\SubmissionDate}{16.10.2025}

\usepackage{caption}
\usepackage{enumitem}
\usepackage{fancyhdr}
\usepackage{float}
\usepackage{fontspec}
\usepackage{geometry}
\usepackage{graphicx}
\usepackage{hyperref}
\usepackage{hyphenat}
\usepackage{indentfirst}
\usepackage{listings}
\usepackage{polyglossia}
\usepackage{setspace}
\usepackage{unicode-math}
\usepackage{xcolor}



\defaultfontfeatures{Ligatures=TeX}
\graphicspath{{./images/}}
\fancyhf{}
\hypersetup{unicode=true, colorlinks=true, linkcolor=black, urlcolor=black}
\pagestyle{fancy}
\setlength{\headheight}{18pt}
\setmainfont{Times New Roman}
\setmathfont{Latin Modern Math}
\setmonofont[Scale=0.8]{DejaVu Sans Mono}
\setotherlanguage{english}
\setsansfont{Arial}
\onehalfspacing

\renewcommand\headrulewidth{0pt}
\cfoot{\thepage}

\geometry{
	a4paper,
	left=25mm,
	right=10mm,
	top=15mm,
	bottom=15mm,
}

\definecolor{codegreen}{rgb}{0,0.6,0}
\definecolor{codegray}{rgb}{0.5,0.5,0.5}
\definecolor{codepurple}{rgb}{0.58,0,0.82}
\definecolor{backcolour}{rgb}{ 0.976, 0.976, 0.976 }

\lstdefinestyle{mystyle}{
	backgroundcolor=\color{backcolour},   
	commentstyle=\color{codegreen},
	keywordstyle=\color{magenta},
	numberstyle=\tiny\color{codegray},
	stringstyle=\color{codepurple},
	basicstyle=\ttfamily\footnotesize,
	breakatwhitespace=false,         
	breaklines=true,                 
	captionpos=b,                    
	keepspaces=true,                 
	numbers=left,                    
	numbersep=5pt,                  
	showspaces=false,                
	showstringspaces=false,
	showtabs=false,                  
	tabsize=2
}

\lstset{style=mystyle}

\setlist[enumerate]{
	itemsep=0.0\baselineskip,
	left=1.25cm,
	rightmargin=10mm,
	labelsep=0.5cm,
	listparindent=1.25cm,
	parsep=0pt
}

\hypersetup{
	colorlinks=true,
	linkcolor=blue,
	filecolor=magenta,      
	urlcolor=cyan
}


\begin{document}
	\tolerance=350 % Or just increase the number
	\emergencystretch=3em
	
	\begin{titlepage}
		\begin{center}
			{Національний технічний університет України\\
				«Київський політехнічний інститут імені Ігоря Сікорського» \\[1.0em] }
			{Факультет інформатики та обчислювальної техніки\\}
			{Кафедра обчислювальної техніки \\[5.0em]}
			
			{\textbf{ЗВІТ}\\[1em]}
			{\textbf{з лабораторних робіт №\LabNumber} \\}
			{\textbf{з дисципліни "\CourseTitle"} \\[2.0em]}
			
			{\textbf{Тема: \Topic} \\[2.0em]}
			
			{\textbf{Варіант №\Variant} \\[5.0em]}
			
			\begin{flushright}
				Виконав: \\
				Студент \CourseNumber{} курсу, групи \StudentGroup \\
				\StudentName \\[2.0em]
			\end{flushright}
			
			\begin{flushright}
				Перевірила: \\
				\Teacher \\[2.0em]
			\end{flushright}
			
			\begin{flushright}
				Дата здачі: \SubmissionDate \\[5.0em]
			\end{flushright}
		
			\vfill
			КИЇВ -- \Year
		\end{center}
	\end{titlepage}
	
	\setlength{\parindent}{1.25cm}
	
	\textbf{Мета роботи.} Визначення множини еквівалентних форм АВ з використанням заданих алгебраїчних законів.
	
	\textbf{Вхідні дані:} Коректний арифметичний вираз в аналітичному вигляді після успішного виконання лексичного та синтаксичного аналізу (результат виконання лабораторної роботи №1) або дерево паралельної форми арифметичного виразу (результат виконання лабораторної роботи №2). Крім того, кожний студент отримує два типи перетворень, які необхідно реалізувати відповідно у лабораторних роботах №3 та №4.
	
	\textbf{Завдання:} Улабораторних роботах №3 та №4 на основі перерахованих вхідних даних розробити алгоритми та програмно реалізувати отримання множини еквівалентних форм АВ або відповідних дерев паралельних форм з урахуванням алгебраїчних законів комутативності, дистрибутивності, асоціативності, а також з розкриття дужок.
	
	\begin{center}
		\textbf{Хід роботи.}
	\end{center}	
	В рамках виконання лабораторної роботи №3-4 було реалізовано механізм пошуку еквівалентних форм арифметичного виразу. Завдання полягало у застосуванні дистрибутивного (розкриття дужок) та асоціативного (винесення спільного множника) законів для генерації множини всіх унікальних форм, що походять від вхідного виразу.
	
	За основу стратегії пошуку було взято двоетапний алгоритм, що відповідає графу, наведеному у методичних матеріалах. Цей підхід чітко розділяє процес на дві фази: розширення та факторинг. Для керування пошуком та уникнення дублікатів і циклів використовується структура даних \texttt{HashSet}, яка зберігає канонічні рядкові представлення кожної вже відвіданої форми. Основний алгоритм пошуку реалізований як пошук у ширину (\texttt{BFS}) з використанням черги \texttt{VecDeque} для кожного етапу, що гарантує покрове знаходження форм.
	
	Перший етап, розширення, починається з оригінального виразу. На кожній ітерації \texttt{BFS} викликається функція \texttt{get\_all\_single\_step\_expansions}. Ця функція рекурсивно сканує поточне дерево \texttt{AST} у пошуках вузлів \texttt{BinaryOperationKind::Multiply}, у яких один з операндів є бінарною операцією \texttt{Plus} або \texttt{Minus}. Це відповідає патернам \texttt{A * (B +/- C)} або \texttt{(A +/- B) * C}. Для кожного знайденого вузла створюється нова копія дерева, де цей вузол замінюється на його розкритий еквівалент, наприклад, \texttt{(A * B) + (A * C)}. Нові згенеровані форми додаються до черги, якщо вони не були відвідані раніше. Цей процес триває, доки черга розширення не вичерпається. Кінцевою точкою цього етапу є форма, в якій більше неможливо застосувати дистрибутивний закон (розкриття дужок множенням), що відповідає Вузлу 7 на графі з пам'ятки.
	
	Важливим проміжним кроком, доданим між двома фазами, є "вирівнювання" (flattening) отриманої повністю розкритої форми. Форма, що є кінцевою для Етапу 1, може все ще містити дужки, наприклад, ... - \texttt{(a * x - t * x)}. Щоб отримати істинно "пласку" форму без дужок, до цього вузла застосовуються функції \texttt{transform\_recursive} та \texttt{fold\_recursive}. Функція \texttt{transform\_recursive} рекурсивно застосовує правила перетворення унарного мінуса, наприклад, \texttt{-(A - B)} на \texttt{(-A) + B}. Одразу після цього \texttt{fold\_recursive} спрощує структуру, наприклад, \texttt{A + (-B)} на \texttt{A - B}. Результатом є нова, канонічна форма (у виводі це Форма 9), яка не містить зайвих дужок і слугує стартовою точкою для наступного етапу. Ця нова форма також додається до загального списку знайдених форм.
	
	Другий етап, факторинг, починає свій пошук BFS саме з цієї нової "пласкої" форми. Логіка асоціативного закону є значно складнішою. Ключовий алгоритм \texttt{get\_all\_possible\_factorings} спершу "вирівнює" дерево додавання/віднімання за допомогою функції \texttt{local\_collect\_operands}. Ця функція коректно обробляє операції \texttt{Minus}, перетворюючи вираз \texttt{A - B} на список доданків, де другий операнд представлений як \texttt{UnaryMinus ((-B))}. Таким чином, вираз \texttt{ak - ck - ax} перетворюється на плаский список \texttt{[ (ak), (-ck), (-ax) ]}. Далі алгоритм ітерує по кожному доданку, знаходячи його множники за допомогою \texttt{get\_factors} (яка коректно працює з унарним мінусом, наприклад, знаходячи \texttt{a} у \texttt{-(a*x)}). Для кожного знайденого фактора (наприклад, \texttt{a}), він шукає цей фактор в інших доданках списку. Залишки збираються за допомогою \texttt{get\_remainder} (яка також обробляє знаки, наприклад, \texttt{get\_remainder(-(a*x), a)} повертає \texttt{(-x)}). Якщо знайдено групу з кількох доданків зі спільним фактором, з їхніх залишків (наприклад, \texttt{[k, (-x)]}) будується нове збалансоване дерево додавання \texttt{k + (-x)}, яке потім множиться на спільний фактор \texttt{a * (k + (-x))}. Цей новий вузол разом з рештою доданків, що не увійшли до групи, формує нову еквівалентну форму, яка додається до черги BFS для подальшого аналізу.
	
	Нарешті, для забезпечення читабельного виводу результатів, було реалізовано окрему функцію \texttt{to\_pretty\_string}. На відміну від \texttt{to\_canonical\_string}, яка використовується для \texttt{HashSet} і містить надлишкові дужки, "красивий" друк рекурсивно обходить дерево, враховуючи пріоритет операцій. Дужки додаються лише тоді, коли пріоритет дочірньої операції нижчий за батьківський (наприклад, \texttt{(a + b) * c}). Крім того, до цієї функції було додано спеціальну логіку для коректного форматування унарних мінусів: конструкції \texttt{AST} виду \texttt{A + (-B)} автоматично перетворюються на рядок \texttt{"A - B"}, а \texttt{(-A) + B} -- на \texttt{"B - A"}. Це дозволило отримати фінальний список з 25 форм, включаючи повністю розкриту форму (Форма 9), у чистому та зрозумілому алгебраїчному вигляді.
	
	\begin{figure}[H]
		\centering
		\includegraphics[height=9cm]{01} 
	\end{figure}
	
	
	\textbf{Висновок.}
	Під час виконання лабораторної роботи №3-4 було успішно розроблено та програмно реалізовано алгоритм для знаходження множини еквівалентних форм арифметичного виразу. Було імплементовано двоетапний підхід, що базується на графовій моделі з методичних матеріалів: спочатку застосовується дистрибутивний закон для повного розкриття дужок (розширення), а потім -- асоціативний закон для винесення спільних множників (факторинг). В результаті роботи програма коректно генерує повний список унікальних, алгебраїчно правильних форм, включаючи повністю "плаский" вираз без дужок, що повністю відповідає меті лабораторної роботи.
		
	\pagebreak
	
	\begin{center}
		\textbf{Лістинг.}
	\end{center}
	
	\textbf{associative.rs}
	
	\begin{lstlisting}[language=Rust]
		use crate::compiler::ast::tree::{
			AbstractSyntaxTree, AstError, AstNode, BinaryOperationKind,
		};
		use std::collections::{HashSet, VecDeque};
		
		impl AbstractSyntaxTree {
			/// Returns a VECTOR of trees, where each tree is ONE step of the factoring.
			pub fn get_all_single_step_factorings(&self) -> Vec<AbstractSyntaxTree> {
				let mut all_new_forms = Vec::new();
				// We start factoring from the root node.
				Self::get_all_possible_factorings(self.peek.clone(), &mut all_new_forms);
				all_new_forms
			}
			
			/// Finds *all* possible *single* factoring steps from the current node.
			fn get_all_possible_factorings(
			node: AstNode, all_forms: &mut Vec<AbstractSyntaxTree>,
			) {
				// We can only factor terms from an addition or subtraction chain.
				match &node {
					AstNode::BinaryOperation { operation, .. }
					if *operation == BinaryOperationKind::Plus
					|| *operation == BinaryOperationKind::Minus =>
					{
						// This is a node we can start collecting from.
					},
					_ => return, // Not an additive/subtractive node, nothing to factor.
				};
				
				// --- STEP 1: Flatten the expression ---
				// This is the CRITICAL FIX. We flatten the tree into a list of terms.
				// `ak - ck - ax` will become `[ (ak), (-ck), (-ax) ]`
				let mut summands = Vec::new();
				Self::local_collect_operands(node, &mut summands);
				
				if summands.len() < 2 {
					return; // Not enough terms to factor.
				}
				
				// --- STEP 2: Find all possible factor groupings ---
				let mut unique_factors: HashSet<String> = HashSet::new();
				
				// Iterate over every term to use as the "base" for a group
				for i in 0..summands.len() {
					let factors_i = Self::get_factors(&summands[i]);
					if factors_i.is_empty() {
						continue; // This term (e.g., 'a') has no factors.
					}
					
					// Iterate over each factor of the base term (e.g., 'a' and 'k' for 'ak')
					for factor in factors_i {
						let tree = AbstractSyntaxTree::from_node(factor);
						let factor_key = tree.to_canonical_string();
						let factor = tree.peek;
						
						// We use a HashSet to avoid generating duplicates.
						if !unique_factors.insert(factor_key) {
							continue;
						}
						
						// Get the remainder for the base term (e.g., 'k' from 'ak' with factor 'a')
						if let Some(remainder_i) = Self::get_remainder(&summands[i], &factor) {
							let mut current_group_terms = vec![remainder_i];
							let mut current_group_indices = vec![i];
							
							// --- STEP 3: Find other terms with the same factor ---
							for (j, summand) in summands.iter().enumerate().skip(i + 1) {
								// e.g., check '-ax' for factor 'a'
								if let Some(remainder_j) = Self::get_remainder(summand, &factor) {
									// Remainder will be '(-x)'. This is correct.
									current_group_terms.push(remainder_j);
									current_group_indices.push(j);
								}
							}
							
							// --- STEP 4: Build the new factored form ---
							if current_group_terms.len() > 1 {
								// We found a group! (e.g., factor 'a' with remainders [k, (-x)])
								let mut new_summands = Vec::new();
								
								// 1. Create the new factored node: a * (k + (-x))
								// We *always* sum the remainders with `Plus`.
								let sum_of_terms = Self::local_build_balanced_tree(
								current_group_terms,
								BinaryOperationKind::Plus,
								)
								.unwrap_or(AstNode::Number(0.0)); // Should not happen
								
								let new_node = AstNode::BinaryOperation {
									operation: BinaryOperationKind::Multiply,
									left: Box::new(factor.clone()),
									right: Box::new(sum_of_terms),
								};
								new_summands.push(new_node);
								
								// 2. Add all the terms that were *not* part of this group
								for (idx, term) in summands.iter().enumerate() {
									if !current_group_indices.contains(&idx) {
										new_summands.push(term.clone());
									}
								}
								
								// 3. Re-assemble the final tree from all terms
								// We *always* build the final tree with `Plus` as well.
								if let Ok(final_node) = Self::local_build_balanced_tree(
								new_summands,
								BinaryOperationKind::Plus,
								) {
									all_forms.push(AbstractSyntaxTree::from_node(final_node));
								}
							}
						}
					}
				}
			}
			
			/// Flattens a chain of `Plus`/`Minus` nodes into a flat Vec of operands.
			/// `A - B` is treated as `A` and `(-B)`.
			/// `ak - ck - ax` -> `[ (ak), (-ck), (-ax) ]`
			fn local_collect_operands(node: AstNode, operands: &mut Vec<AstNode>) {
				match node {
					// Case: A + B
					AstNode::BinaryOperation {
						operation: BinaryOperationKind::Plus,
						left,
						right,
					} => {
						// Collect operands from both sides
						Self::local_collect_operands(*left, operands);
						Self::local_collect_operands(*right, operands);
					},
					// Case: A - B
					AstNode::BinaryOperation {
						operation: BinaryOperationKind::Minus,
						left,
						right,
					} => {
						// Collect operands from the left side
						Self::local_collect_operands(*left, operands);
						// Collect operands from the right side, but wrap them in UnaryMinus
						Self::local_collect_operands_with_minus(*right, operands);
					},
					// Base case: A standalone term (like 'ak' or '-ck')
					_ => {
						operands.push(node);
					},
				}
			}
			
			/// Helper for `local_collect_operands` to correctly apply UnaryMinus.
			/// This handles `-(A+B) -> -A + -B` and `-(-A) -> A`.
			fn local_collect_operands_with_minus(node: AstNode, operands: &mut Vec<AstNode>) {
				match node {
					// Case: -(-A) => A
					AstNode::UnaryOperation {
						operation: crate::compiler::ast::tree::UnaryOperationKind::Minus,
						expression,
					} => {
						Self::local_collect_operands(*expression, operands);
					},
					// Case: -(A + B) => -A, -B
					AstNode::BinaryOperation {
						operation: BinaryOperationKind::Plus,
						left,
						right,
					} => {
						Self::local_collect_operands_with_minus(*left, operands);
						Self::local_collect_operands_with_minus(*right, operands);
					},
					// Case: -(A - B) => -A, B
					AstNode::BinaryOperation {
						operation: BinaryOperationKind::Minus,
						left,
						right,
					} => {
						Self::local_collect_operands_with_minus(*left, operands);
						Self::local_collect_operands(*right, operands); // Note: right side becomes positive
					},
					// Base case: -(term) => push(-term)
					_ => {
						operands.push(AstNode::UnaryOperation {
							operation: crate::compiler::ast::tree::UnaryOperationKind::Minus,
							expression: Box::new(node),
						});
					},
				}
			}
			
			/// Builds a balanced tree from a list of operands.
			/// (This is a standard helper function, same as in balancer.rs)
			fn local_build_balanced_tree(
			operands: Vec<AstNode>, op_kind: BinaryOperationKind,
			) -> Result<AstNode, AstError> {
				if operands.is_empty() {
					return Err(AstError::CannotBuildEmptyTree);
				}
				let mut queue: VecDeque<AstNode> = operands.into();
				while queue.len() > 1 {
					let level_size = queue.len();
					for _ in 0..(level_size / 2) {
						let left = queue.pop_front().ok_or(AstError::FailedPopFromQueue)?;
						let right = queue.pop_front().ok_or(AstError::FailedPopFromQueue)?;
						let new_node = AstNode::BinaryOperation {
							operation: op_kind.clone(),
							left: Box::new(left),
							right: Box::new(right),
						};
						queue.push_back(new_node);
					}
					if !level_size.is_multiple_of(2) {
						let odd_one_out =
						queue.pop_front().ok_or(AstError::FailedPopFromQueue)?;
						queue.push_back(odd_one_out);
					}
				}
				queue.pop_front().ok_or(AstError::FailedPopFromQueue)
			}
			
			/// Helper function: finds the factors for a term.
			/// `A * B` -> `[A, B]`
			/// `-(A * B)` -> `[A, B]` (sign is handled by get_remainder)
			fn get_factors(node: &AstNode) -> Vec<AstNode> {
				match node {
					// Case: A * B
					AstNode::BinaryOperation {
						operation: BinaryOperationKind::Multiply,
						left,
						right,
					} => {
						vec![*left.clone(), *right.clone()]
					},
					// Case: -(A * B)
					AstNode::UnaryOperation {
						operation: crate::compiler::ast::tree::UnaryOperationKind::Minus,
						expression,
					} => {
						if let AstNode::BinaryOperation {
							operation: BinaryOperationKind::Multiply,
							left,
							right,
						} = expression.as_ref()
						{
							// Factors are A and B. The minus sign is part of the "remainder".
							vec![*left.clone(), *right.clone()]
						} else {
							vec![]
						}
					},
					_ => vec![], // Not a multiplication, no factors
				}
			}
			
			/// Helper function: for `node` (term) and `factor`, returns the remainder.
			/// `(A * B, A)` -> `B`
			/// `(-(A * B), A)` -> `(-B)`
			fn get_remainder(node: &AstNode, factor: &AstNode) -> Option<AstNode> {
				match node {
					// Case: A * B
					AstNode::BinaryOperation {
						operation: BinaryOperationKind::Multiply,
						left,
						right,
					} => {
						if left.as_ref() == factor {
							return Some(*right.clone()); // (A * B) / A = B
						}
						if right.as_ref() == factor {
							return Some(*left.clone()); // (A * B) / B = A
						}
						None
					},
					// Case: -(A * B)
					AstNode::UnaryOperation {
						operation: op @ crate::compiler::ast::tree::UnaryOperationKind::Minus,
						expression,
					} => {
						if let AstNode::BinaryOperation {
							operation: BinaryOperationKind::Multiply,
							left,
							right,
						} = expression.as_ref()
						{
							if left.as_ref() == factor {
								// -(A * B) / A = -B
								return Some(AstNode::UnaryOperation {
									operation: op.clone(),
									expression: right.clone(),
								});
							}
							if right.as_ref() == factor {
								// -(A * B) / B = -A
								return Some(AstNode::UnaryOperation {
									operation: op.clone(),
									expression: left.clone(),
								});
							}
						}
						None
					},
					_ => None,
				}
			}
		}
		
	\end{lstlisting}
	
	\textbf{distributive.rs}
	
	\begin{lstlisting}[language=Rust]
		use crate::compiler::ast::tree::{AbstractSyntaxTree, AstNode, BinaryOperationKind};
		
		// A "path" is a sequence of 0s and 1s (and 2s for Unary)
		// 0 = left child, 1 = right child, 2 = unary expression
		type NodePath = Vec<u8>;
		
		impl AbstractSyntaxTree {
			/// Returns a vector of AbstractSyntaxTree, each representing a single-step expansion.
			pub fn get_all_single_step_expansions(&self) -> Vec<AbstractSyntaxTree> {
				let mut expandable_nodes_paths: Vec<NodePath> = Vec::new();
				
				// 1. Find all paths to nodes that can be expanded
				Self::find_expandable_nodes_recursive(
				&self.peek,
				&mut vec![],
				&mut expandable_nodes_paths,
				);
				
				let mut all_forms = Vec::new();
				
				// 2. For each found path...
				for path in expandable_nodes_paths {
					// Create a fresh copy of the tree for this single expansion
					let mut new_tree = self.clone();
					
					// 3. Get a mutable reference to the node at that path
					if let Some(target_node) =
					Self::get_node_mut_by_path(&mut new_tree.peek, &path)
					{
						// 4. Replace that node with its expanded version
						// `perform_expansion` is guaranteed to expand *this* node
						*target_node = Self::perform_expansion(target_node.clone());
					}
					all_forms.push(new_tree);
				}
				
				all_forms
			}
			
			/// Recursively finds nodes matching the pattern `A * (B +/- C)` or `(A +/- B) * C`
			fn find_expandable_nodes_recursive(
			node: &AstNode, path: &mut NodePath, paths: &mut Vec<NodePath>,
			) {
				if let AstNode::BinaryOperation {
					operation: BinaryOperationKind::Multiply,
					left,
					right,
				} = node
				{
					// Pattern 1: (A +/- B) * C
					if let AstNode::BinaryOperation { operation: op, .. } = left.as_ref()
					&& (*op == BinaryOperationKind::Plus || *op == BinaryOperationKind::Minus)
					{
						// This `Multiply` node can be expanded. Save its path.
						paths.push(path.clone());
					}
					// Pattern 2: A * (B +/- C)
					if let AstNode::BinaryOperation { operation: op, .. } = right.as_ref()
					&& (*op == BinaryOperationKind::Plus || *op == BinaryOperationKind::Minus)
					{
						// This `Multiply` node can also be expanded. Save its path.
						paths.push(path.clone());
					}
				}
				
				// Recursive traversal to check children
				match node {
					AstNode::BinaryOperation { left, right, .. } => {
						path.push(0); // 0 = left
						Self::find_expandable_nodes_recursive(left, path, paths);
						path.pop();
						
						path.push(1); // 1 = right
						Self::find_expandable_nodes_recursive(right, path, paths);
						path.pop();
					},
					AstNode::UnaryOperation { expression, .. } => {
						path.push(2); // 2 = expression
						Self::find_expandable_nodes_recursive(expression, path, paths);
						path.pop();
					},
					// Base cases (Number, Identifier): nowhere else to go
					_ => {},
				}
			}
			
			/// Helper function to get mutable reference to a node by path
			pub fn get_node_mut_by_path<'a>(
			node: &'a mut AstNode, path: &[u8],
			) -> Option<&'a mut AstNode> {
				let mut current = node;
				for &index in path {
					match current {
						AstNode::BinaryOperation { left, right, .. } => {
							current = if index == 0 { left } else { right };
						},
						AstNode::UnaryOperation { expression, .. } => {
							current = expression;
						},
						_ => return None, // Path is invalid
					}
				}
				Some(current)
			}
			
			/// Performs ONE unfolding on a node that is *guaranteed* to be `Multiply`
			/// and have at least one child that is `Plus` or `Minus`.
			fn perform_expansion(node: AstNode) -> AstNode {
				if let AstNode::BinaryOperation {
					operation: BinaryOperationKind::Multiply,
					left,
					right,
				} = node
				{
					// Case 1: A * (B +/- C)
					if let AstNode::BinaryOperation {
						operation: op @ (BinaryOperationKind::Plus | BinaryOperationKind::Minus),
						left: b,  // B
						right: c, // C
					} = *right
					{
						// Create (A * B)
						let new_left = AstNode::BinaryOperation {
							operation: BinaryOperationKind::Multiply,
							left: left.clone(), // A
							right: b,
						};
						// Create (A * C)
						let new_right = AstNode::BinaryOperation {
							operation: BinaryOperationKind::Multiply,
							left, // A
							right: c,
						};
						// Return (A*B) +/- (A*C)
						return AstNode::BinaryOperation {
							operation: op,
							left: Box::new(new_left),
							right: Box::new(new_right),
						};
					}
					
					// Case 2: (A +/- B) * C
					if let AstNode::BinaryOperation {
						operation: op @ (BinaryOperationKind::Plus | BinaryOperationKind::Minus),
						left: a,  // A
						right: b, // B
					} = *left
					{
						// Create (A * C)
						let new_left = AstNode::BinaryOperation {
							operation: BinaryOperationKind::Multiply,
							left: a,
							right: right.clone(), // C
						};
						// Create (B * C)
						let new_right = AstNode::BinaryOperation {
							operation: BinaryOperationKind::Multiply,
							left: b,
							right, // C
						};
						// Return (A*C) +/- (B*C)
						return AstNode::BinaryOperation {
							operation: op,
							left: Box::new(new_left),
							right: Box::new(new_right),
						};
					}
					
					// If patterns didn't match (e.g., this was called on a wrong node)
					// return the original node.
					return AstNode::BinaryOperation {
						operation: BinaryOperationKind::Multiply,
						left,
						right,
					};
				}
				
				// Return the node as is if it's not `Multiply`
				node
			}
		}
	\end{lstlisting}
	
	\textbf{equivalent\_forms.rs}
	
	\begin{lstlisting}[language=Rust]
		use crate::compiler::ast::tree::AbstractSyntaxTree;
		use crate::compiler::reports::Reporter;
		use crate::utils::StringBuffer;
		use std::collections::{HashSet, VecDeque};
		
		impl AbstractSyntaxTree {
			pub fn find_equivalent_forms(&self) -> Vec<AbstractSyntaxTree> {
				let mut all_forms: Vec<AbstractSyntaxTree> = Vec::new();
				let mut visited: HashSet<String> = HashSet::new();
				let mut expansion_queue: VecDeque<AbstractSyntaxTree> = VecDeque::new();
				
				// --- Stage 1: Expansion (Distributive Law, Nodes 0-7 from memo) ---
				// This stage finds all forms *only* by expanding parentheses.
				// It performs a Breadth-First Search (BFS) starting from the original expression.
				
				let initial_key = self.to_canonical_string();
				expansion_queue.push_back(self.clone());
				visited.insert(initial_key);
				all_forms.push(self.clone());
				
				let mut fully_expanded_node: Option<AbstractSyntaxTree> = None;
				
				while let Some(current_ast) = expansion_queue.pop_front() {
					// Get all possible next forms by applying *one step* of expansion
					let expansion_steps = current_ast.get_all_single_step_expansions();
					
					// If a node has no possible expansions, it's a "leaf" in this stage.
					// We assume the first one we find is the fully expanded form (Node 7).
					if expansion_steps.is_empty() && fully_expanded_node.is_none() {
						fully_expanded_node = Some(current_ast.clone());
					}
					
					for expanded_ast in expansion_steps {
						let key = expanded_ast.to_canonical_string();
						if !visited.contains(&key) {
							visited.insert(key.clone());
							all_forms.push(expanded_ast.clone());
							expansion_queue.push_back(expanded_ast);
						}
					}
				}
				
				// --- Stage 1.5: Flattening ---
				// We take the "fully expanded" node (Node 7) and apply unary minus
				// rules like `-(A-B) -> -A+B` to get the *truly* flat form.
				// This is the form you're looking for, which has no parentheses.
				
				let Some(node_to_flatten) = fully_expanded_node else {
					log::warn!(
					"No fully expanded form (Node 7) found. Skipping Stage 1.5 and 2."
					);
					return all_forms;
				};
				
				let node_to_flatten_copy = node_to_flatten.clone();
				let start_node_for_factoring =
				match Self::transform_recursive(node_to_flatten.peek)
				.and_then(Self::fold_recursive)
				{
					Ok(flattened_node_peek) => {
						let flattened_ast =
						AbstractSyntaxTree::from_node(flattened_node_peek);
						let key = flattened_ast.to_canonical_string();
						if !visited.contains(&key) {
							// Add this new, truly flat form if it's unique
							visited.insert(key);
							all_forms.push(flattened_ast.clone());
						}
						flattened_ast // This is the new starting point for factoring
					},
					Err(e) => {
						log::error!("Failed to flatten Node 7: {:?}. Using un-flattened.", e);
						node_to_flatten_copy // Fallback to the un-flattened node
					},
				};
				
				// --- Stage 2: Factoring (Associative Law, Nodes 7-13) ---
				// This stage now starts from the *truly flat* form.
				
				let mut factoring_queue: VecDeque<AbstractSyntaxTree> = VecDeque::new();
				factoring_queue.push_back(start_node_for_factoring);
				// We don't need to re-add to `visited` or `all_forms`,
				// as it was handled in Stage 1.5
				
				while let Some(current_ast) = factoring_queue.pop_front() {
					let factoring_steps = current_ast.get_all_single_step_factorings();
					
					for factored_ast in factoring_steps {
						let key = factored_ast.to_canonical_string();
						if !visited.contains(&key) {
							visited.insert(key.clone());
							all_forms.push(factored_ast.clone());
							factoring_queue.push_back(factored_ast);
						}
					}
				}
				
				all_forms
			}
		}
		
		impl Reporter {
			pub fn finding_equivalent_form(&self, forms: &[String]) -> String {
				let mut buffer = StringBuffer::default();
				
				buffer.add_line(format!("Found {} equivalent forms!\n", forms.len()));
				
				for (index, form) in forms.iter().enumerate() {
					buffer.add_line(format!("{}) {}", index + 1, form));
				}
				
				buffer.get()
			}
		}
	\end{lstlisting}
\end{document}