\documentclass[14pt, a4paper]{extreport}

\usepackage{kpi-lab}
% Defined by subject
\newcommand{\CourseTitle}{Сучасні мобільні операційні системи}
\newcommand{\Teacher}{асистент, Нестерук Андрій Олександрович}
\newcommand{\ReportYear}{2026}
% Defined by Student
\newcommand{\StudentName}{Ковальов Олександр}
\newcommand{\StudentGroup}{ІМ-51мн}
\newcommand{\Variant}{4}
\newcommand{\CourseNumber}{1}
% Defined by Submission
\newcommand{\Topic}{Робота з мультимедійними файлами}
\newcommand{\LabNumber}{4}

\begin{document}
	\pagenumbering{gobble}
	
	\begin{titlepage}
		\begin{center}
			\normalsize
			Національний технічний університет України\\
			«Київський політехнічний інститут імені Ігоря Сікорського» \\[0.5em]
			Факультет інформатики та обчислювальної техніки\\
			Кафедра обчислювальної техніки
			
			\vfill
			
			{\large \textbf{ЗВІТ}\\[0.5em]}
			{з лабораторної роботи №\LabNumber \\}
			{з дисципліни <<\CourseTitle>>}
			
			\vspace{2em}
			
			{\textbf{Тема: <<\Topic>>}}
			
			\vspace{1em}
			
			{\textbf{Варіант №\Variant}}
			
			\vfill
			
			\begin{flushright}
				Виконав: \\ студент \CourseNumber{} курсу, групи \StudentGroup \\
				\StudentName \\[1em]
				Перевірив: \\ \Teacher
			\end{flushright}
			
			\vfill
			
			\begin{flushright}
				Дата здачі: \DTMtoday
			\end{flushright}
			
			\vspace{5em}
			КИЇВ -- \ReportYear
		\end{center}
	\end{titlepage}
	
	\clearpage
	\pagenumbering{arabic}
	\setcounter{page}{2}
	
	\subsection*{Мета роботи} 
	Вивчити роботу з потоками, навчитися працювати з мультимедійними файлами та з класом \texttt{AsyncTask}.
	
	\subsection*{Завдання 1}
	Розгляньте приклад передачі даних.
	
	\begin{minted}{java} 
		MyAsyncTask at = new MyAsyncTask();
		at.execute("url1", "url2");
		doInBackground(String ... urls)
	\end{minted}
	
	\subsection*{Завдання 2}
	Розгляньте приклад виведення проміжних даних.
	
	\begin{minted}{java} 
		@Override
		protected void doInBackground (String ... urls) {
			try {
				int cnt = 0; 
				for (String url: urls) {
					// обробляємо перший параметр
					...
					// виводимо проміжні результати
					cnt++;
					publishProgress(cnt);
				}
				
				TimeUnit.SECONDS.sleep (1);
			} catch (InterruptedException e) {
				e.printStackTrace ();
			}
			return null;
		}
		
		@Override
		protected void onProgressUpdate(Integer ... values) {
			super.onProgressUpdate (values); 
			tv.setText("оброблено" + values[0] + "параметрів");
		}
	\end{minted}
	
	\subsection*{Завдання 3}
	Перевірте приклад створення простої асинхронної \texttt{Activity}.
	
	\begin{minted}{java}
		public class MainActivity extends Activity {
			MyAsyncTask at;
			TextView tv;
			
			public void onCreate (Bundle savedInstance State) {
				super.onCreate(savedInstanceState);
				setContentView (R.layout.main);
				
				tv = (TextView) findViewById(R.id.tv);
				MyAsyncTask at = new MyAsyncTask();
				at.execute ();
			}
		}
	\end{minted}
	
	\subsection*{Завдання 4}
	На підставі вивчених прикладів розробити додаток, що зберігає статистику пісень, які програються на WebRadio. Для збереження пісні і назви необхідно створити базу даних, що містить таблицю з наступними полями:
	
	\begin{enumerate}
		\item ID
		\item Виконавець
		\item Назва треку
		\item Час внесення запису
	\end{enumerate}
	
	При запуску програми необхідно проводити перевірку підключення до Інтернету. У разі якщо сигнал мережі переривається -- виводити спливаюче повідомлення (\texttt{Toast}) з попередженням про запуск в автономному режимі (доступний тільки перегляд внесених раніше записів). Після включення додаток повинен робити асинхронне опитування сервера з інтервалом 20 секунд. Якщо назва треку не збігається з останнім записом в таблиці необхідно зробити запис в базі даних. URL адреси, за якою можна отримати інформацію про поточний трек і виконавця: \url{https://webradio.io/api/radio/pi/current-song}
	
	У разі успішного виконання запиту результат буде мати вигляд:
	
	\begin{minted}{json}
		{ "Result": "success", "info": "Виконавець - Назва треку"}
	\end{minted}
	
	У разі помилки API поверне наступний рядок:
	
	\begin{minted}{json}
		{ "Result": "error", "info": "Інформація про помилку"}
	\end{minted}
	
	Програма повинна дозволяти переглядати внесені в базу даних записи.
	
	\section*{Хід роботи.}
	
	\textbf{Завдання 1.} У першому завданні продемонстровано механізм передачі вхідних параметрів до асинхронного завдання. Спочатку створюється екземпляр класу \texttt{MyAsyncTask}, після чого викликається метод \texttt{execute}, у який передаються необхідні аргументи, наприклад, рядки з веб-адресами. Ці передані значення автоматично потрапляють до системного методу \texttt{doInBackground}, де вони приймаються у вигляді масиву змінної довжини завдяки синтаксису \texttt{String... urls}, що дозволяє фоновому потоку отримати та почати обробку цієї інформації.
	
	\textbf{Завдання 2.} Друге завдання ілюструє процес взаємодії фонового потоку з головним потоком інтерфейсу користувача для відображення проміжних результатів. У методі \texttt{doInBackground} відбувається ітерація по масиву переданих параметрів, і після обробки кожного елемента лічильник збільшується та передається у спеціальний метод \texttt{publishProgress}. Цей виклик сигналізує системі та автоматично ініціює спрацювання методу \texttt{onProgressUpdate}, який працює вже в головному потоці програми (UI Thread). Це забезпечує можливість безпечно оновити графічні компоненти, наприклад, змінити текст у \texttt{TextView} та показати поточну кількість оброблених даних без ризику блокування або збою інтерфейсу.
	
	\textbf{Завдання 3.} У третьому завданні наведено базову структуру активності, яка ініціює виконання асинхронної операції відразу під час свого створення. У методі \texttt{onCreate} відбувається стандартне налаштування екрана: виклик батьківського методу, встановлення розмітки через \texttt{setContentView} та ініціалізація текстового поля за допомогою \texttt{findViewById}. Одразу після підготовки графічного інтерфейсу створюється об'єкт власного асинхронного класу \texttt{MyAsyncTask} та викликається його метод \texttt{execute} без параметрів, що запускає виконання прописаної фонової логіки паралельно з роботою активності.
	
	\textbf{Завдання 4.} Для початку, так як застосунок потребує підключення до інтернету та перевірку зміни його стану, треба видати відповідні дозволи. Їх треба прописати в \texttt{AndroidManifest.xml}, перед тегом \texttt{application}:
	
	\image{4cm}{permissions}
	
	Після налаштування дозволів було розроблено клас помічника бази даних, який успадковує \texttt{SQLiteOpenHelper}. У ньому визначено структуру таблиці для збереження історії відтворення, яка складається з полів ідентифікатора, виконавця, назви треку та точного часу додавання запису. Для повноцінної роботи з цією таблицею реалізовано три основні методи. Перший відповідає за безпосереднє збереження нового треку. Другий метод виконує пошук та повертає назву останньої збереженої пісні, що критично важливо для порівняння та уникнення запису дублікатів під час постійного опитування сервера. Третій метод зчитує всю таблицю та формує з неї єдиний текстовий рядок для зручного виведення на екран історії.
	
	Основна бізнес-логіка програми зосереджена у класі головної активності. Під час ініціалізації додаток звертається до системного сервісу \texttt{ConnectivityManager}, щоб перевірити наявність активного підключення до мережі Інтернет. Залежно від статусу з'єднання, користувач отримує спливаюче повідомлення про роботу в онлайн-режимі або попередження про перехід в автономний стан. Якщо інтернет доступний, програма запускає асинхронний процес опитування сервера. Замість застарілого класу \texttt{AsyncTask} для цього використано корутини (\texttt{Coroutines}), що дозволило створити безпечний фоновий цикл, який призупиняє своє виконання на 20 секунд між ітераціями, абсолютно не блокуючи графічний інтерфейс користувача.
	
	Під час кожної ітерації фоновий потік надсилає HTTP-запит до API радіостанції та отримує текстову відповідь. Далі ця відповідь обробляється за допомогою класу JSONObject, з якого вилучаються значення за ключами \texttt{artist} та \texttt{title}. Оскільки оновлення інтерфейсу дозволено лише з головного потоку, отримані дані передаються туди за допомогою контексту \texttt{Dispatchers.Main}, після чого на екрані з'являється інформація про поточну пісню. Одразу після цього назва отриманого треку порівнюється з останнім записом у локальній базі даних. Якщо виявляється розбіжність, програма генерує поточну мітку часу та ініціює збереження нової композиції.
	
	\image{5cm}{playing-now}
	
	Для реалізації вимоги щодо перегляду історії було створено окрему активність. Її графічна розмітка є максимально спрощеною і складається з єдиного текстового поля, поміщеного всередину контейнера \texttt{ScrollView}. Таке рішення гарантує можливість вільного прокручування тексту, якщо кількість збережених пісень перевищить розмір екрана. Під час відкриття цього вікна програма звертається до методу бази даних для отримання всього сформованого тексту історії та призначає його відповідному елементу інтерфейсу.
	
	\image{7.5cm}{history}
	
	\subsection*{Висновок}
	
	Під час виконання лабораторної роботи було практично досліджено принципи асинхронного програмування в операційній системі Android. На основі аналізу прикладів із застарілим класом \texttt{AsyncTask} було здійснено перехід до сучасних інструментів, зокрема \texttt{Kotlin Coroutines}, що дозволило реалізувати безпечне виконання тривалих мережевих запитів у фоновому потоці без блокування інтерфейсу користувача. Було розроблено повноцінний додаток для моніторингу поточного треку радіостанції через веб-API, який успішно виконує HTTP-запити, розбирає отримані дані у форматі JSON та зберігає їхню історію у локальній базі даних SQLite з попередньою перевіркою на дублікати. Крім того, засвоєно навички роботи із системними сервісами для перевірки наявності інтернет-з'єднання та інформування користувача про поточний стан за допомогою спливаючих повідомлень \texttt{Toast}.
	
	\section*{Контрольні запитання.}
	
	\begin{enumerate}
		\item \textbf{Як за допомогою класу \texttt{Toast} створити спливаюче повідомлення?} \\ Спливаюче повідомлення створюється шляхом виклику статичного методу \texttt{makeText} класу \texttt{Toast}, якому передаються три параметри: контекст додатку або активності, текст самого повідомлення та тривалість показу (наприклад, \texttt{Toast.LENGTH\_SHORT} або \texttt{Toast.LENGTH\_LONG}). Після формування об'єкта необхідно обов'язково викликати метод \texttt{show()}, щоб повідомлення з'явилося на екрані.
		
		\item \textbf{У яких випадках необхідне логування?} \\ Логування є критично необхідним на етапі розробки та дебагу додатку для відстеження потоку виконання програми, аналізу значень змінних у різних станах та виявлення прихованих помилок. Воно дозволяє розробнику фіксувати моменти виникнення виняткових ситуацій (наприклад, відсутність мережі або помилки парсингу) без переривання роботи самої програми та без виведення незрозумілої технічної інформації на екран кінцевого користувача.
		
		\item \textbf{Що являє собою вікно \texttt{LogCat}? Які існують рівні логування?} \\ Вікно \texttt{LogCat} -- це вбудований інструмент середовища розробки Android Studio, який у реальному часі відображає системні повідомлення, помилки та власні текстові записи, згенеровані програмою. Існує декілька рівнів логування, які класифікують повідомлення за їхньою важливістю: \texttt{Error} (помилки, що призводять до збоїв), \texttt{Warning} (попередження про потенційні проблеми), \texttt{Info} (загальна інформаційна статистика), \texttt{Debug} (детальні дані для дебагу) та \texttt{Verbose} (максимально детальний рівень для глибокого аналізу).
		
		\item \textbf{Як програмно реалізувати логування?} \\ Для програмної реалізації логування використовується системний клас \texttt{Log}, який надає статичні методи, що відповідають кожному рівню деталізації (наприклад, \texttt{Log.e()} для помилок, \texttt{Log.d()} для дебагу, \texttt{Log.i()} для інформації). Кожен такий метод приймає два основні рядкові параметри: тег (TAG), який зазвичай містить назву класу для зручної фільтрації у вікні \texttt{LogCat}, та безпосередньо текст самого повідомлення, яке потрібно зафіксувати.
		
		\item \textbf{Як створити нове Аctivity (опишіть роботу з java-класом, layout-файлом, файлом конфігурації \texttt{AndroidManifest.xml})?} \\ Процес створення нової активності складається з трьох обов'язкових етапів. Спочатку у папці ресурсів \texttt{res/layout} створюється XML-файл розмітки, який визначає візуальну складову вікна. Потім створюється клас, який успадковується від \texttt{AppCompatActivity} чи базового класу \texttt{Activity}, де у перевизначеному методі \texttt{onCreate} відбувається прив'язка створеної розмітки за допомогою виклику \texttt{setContentView}. На фінальному етапі нову активність необхідно обов'язково зареєструвати у файлі \texttt{AndroidManifest.xml}, додавши тег \texttt{<activity>} з вказівкою імені класу всередині блоку \texttt{<application>}, інакше при спробі її відкрити програма аварійно завершить роботу.
		
		\item \textbf{Для чого використовується контекст додатку \texttt{Context}?} \\ Контекст (\texttt{Context}) є базовим інтерфейсом, який надає доступ до глобальної інформації про середовище додатка. Він використовується операційною системою як своєрідний паспорт програми, необхідний для отримання доступу до ресурсів (рядків, зображень, баз даних), ініціалізації системних сервісів (наприклад, для перевірки стану мережі), створення графічних елементів інтерфейсу, запуску нових активностей та показу спливаючих повідомлень.
	\end{enumerate}
	
	\pagebreak
	
	\section*{Лістинг.}
	
	\textbf{DatabaseHelper.kt}
	
	\begin{minted}{kotlin} 
		package ua.kpi.lab4
		
		import android.content.ContentValues
		import android.content.Context
		import android.database.sqlite.SQLiteDatabase
		import android.database.sqlite.SQLiteOpenHelper
		
		// Helper class for managing WebRadio database
		class DatabaseHelper(context: Context) :
		SQLiteOpenHelper(context, DATABASE_NAME, null, DATABASE_VERSION) {
			
			companion object {
				private const val DATABASE_NAME = "webradio.db"
				private const val DATABASE_VERSION = 1
				
				private const val TABLE_NAME = "songs_history"
				private const val COLUMN_ID = "id"
				private const val COLUMN_ARTIST = "artist"
				private const val COLUMN_TRACK = "track_name"
				private const val COLUMN_TIME = "time_added"
			}
			
			override fun onCreate(db: SQLiteDatabase) {
				// Create table for storing song statistics
				val createTableQuery = ("CREATE TABLE $TABLE_NAME ("
				+ "$COLUMN_ID INTEGER PRIMARY KEY AUTOINCREMENT, "
				+ "$COLUMN_ARTIST TEXT, "
				+ "$COLUMN_TRACK TEXT, "
				+ "$COLUMN_TIME TEXT)")
				db.execSQL(createTableQuery)
			}
			
			override fun onUpgrade(db: SQLiteDatabase, oldVersion: Int, newVersion: Int) {
				db.execSQL("DROP TABLE IF EXISTS $TABLE_NAME")
				onCreate(db)
			}
			
			// Insert a new song record into the database
			fun insertSong(artist: String, track: String, time: String) {
				val db = this.writableDatabase
				val values = ContentValues()
				values.put(COLUMN_ARTIST, artist)
				values.put(COLUMN_TRACK, track)
				values.put(COLUMN_TIME, time)
				db.insert(TABLE_NAME, null, values)
				db.close()
			}
			
			// Retrieve the name of the last saved track to prevent duplicates
			fun getLastTrackName(): String? {
				val db = this.readableDatabase
				var lastTrack: String? = null
				val query = "SELECT $COLUMN_TRACK FROM $TABLE_NAME ORDER BY $COLUMN_ID DESC LIMIT 1"
				val cursor = db.rawQuery(query, null)
				
				if (cursor.moveToFirst()) {
					lastTrack = cursor.getString(0)
				}
				cursor.close()
				db.close()
				return lastTrack
			}
			
			// Retrieve all saved songs as a formatted string
			fun getAllSongs(): String {
				val db = this.readableDatabase
				val cursor = db.rawQuery("SELECT * FROM $TABLE_NAME ORDER BY $COLUMN_ID DESC", null)
				val stringBuilder = StringBuilder()
				
				if (cursor.moveToFirst()) {
					do {
						val artist = cursor.getString(cursor.getColumnIndexOrThrow(COLUMN_ARTIST))
						val track = cursor.getString(cursor.getColumnIndexOrThrow(COLUMN_TRACK))
						val time = cursor.getString(cursor.getColumnIndexOrThrow(COLUMN_TIME))
						stringBuilder.append("Час: $time\nВиконавець: $artist\nТрек: $track\n\n")
					} while (cursor.moveToNext())
				} else {
					stringBuilder.append("Історія порожня.")
				}
				cursor.close()
				db.close()
				return stringBuilder.toString()
			}
		}
	\end{minted}
	
	\textbf{MainActivity.kt}
	
	\begin{minted}{kotlin} 
		package ua.kpi.lab4
		
		import android.content.Intent
		import android.net.ConnectivityManager
		import android.net.NetworkCapabilities
		import android.os.Bundle
		import android.widget.Button
		import android.widget.TextView
		import android.widget.Toast
		import androidx.appcompat.app.AppCompatActivity
		import androidx.lifecycle.lifecycleScope
		import kotlinx.coroutines.Dispatchers
		import kotlinx.coroutines.delay
		import kotlinx.coroutines.isActive
		import kotlinx.coroutines.launch
		import kotlinx.coroutines.withContext
		import org.json.JSONObject
		import java.io.BufferedReader
		import java.io.InputStreamReader
		import java.net.HttpURLConnection
		import java.net.URL
		import java.text.SimpleDateFormat
		import java.util.Date
		import java.util.Locale
		
		class MainActivity : AppCompatActivity() {
			
			private lateinit var dbHelper: DatabaseHelper
			private lateinit var tvCurrentStatus: TextView
			
			override fun onCreate(savedInstanceState: Bundle?) {
				super.onCreate(savedInstanceState)
				setContentView(R.layout.activity_main)
				
				dbHelper = DatabaseHelper(this)
				tvCurrentStatus = findViewById(R.id.tvCurrentStatus)
				val btnViewHistory = findViewById<Button>(R.id.btnViewHistory)
				
				btnViewHistory.setOnClickListener {
					startActivity(Intent(this, HistoryActivity::class.java))
				}
				
				// Check internet connection on startup
				if (isNetworkAvailable()) {
					Toast.makeText(this, "Онлайн режим. Опитування сервера...", Toast.LENGTH_SHORT).show()
					startPollingServer()
				} else {
					Toast.makeText(
					this,
					"Автономний режим. Доступний лише перегляд бази.",
					Toast.LENGTH_LONG
					).show()
					tvCurrentStatus.text = "Немає підключення до мережі"
				}
			}
			
			// Helper method to check internet connectivity
			private fun isNetworkAvailable(): Boolean {
				val connectivityManager = getSystemService(CONNECTIVITY_SERVICE) as ConnectivityManager
				val network = connectivityManager.activeNetwork ?: return false
				val activeNetwork = connectivityManager.getNetworkCapabilities(network) ?: return false
				return activeNetwork.hasCapability(NetworkCapabilities.NET_CAPABILITY_INTERNET)
			}
			
			// Coroutine to poll the server every 20 seconds
			private fun startPollingServer() {
				lifecycleScope.launch(Dispatchers.IO) {
					while (isActive) {
						if (isNetworkAvailable()) {
							fetchCurrentSong()
						}
						// Delay for 20 seconds
						delay(20000)
					}
				}
			}
			
			// Fetch data from API and process it
			private suspend fun fetchCurrentSong() {
				try {
					val url = URL("https://webradio.io/api/radio/pi/current-song")
					val connection = url.openConnection() as HttpURLConnection
					connection.requestMethod = "GET"
					connection.connectTimeout = 5000
					
					if (connection.responseCode == HttpURLConnection.HTTP_OK) {
						val reader = BufferedReader(InputStreamReader(connection.inputStream))
						val response = StringBuilder()
						var line: String?
						while (reader.readLine().also { line = it } != null) {
							response.append(line)
						}
						reader.close()
						
						parseAndSaveData(response.toString())
					}
					connection.disconnect()
				} catch (e: Exception) {
					e.printStackTrace()
					updateUIStatus("Помилка з'єднання з сервером")
				}
			}
			
			// Parse JSON and interact with database
			private suspend fun parseAndSaveData(jsonString: String) {
				try {
					val jsonObject = JSONObject(jsonString)
					
					// Fetching data
					val artist = jsonObject.optString("artist", "Невідомий виконавець")
					val track = jsonObject.optString("title", "Невідомий трек")
					
					updateUIStatus("Зараз грає:\n$artist - $track")
					
					val lastTrackInDb = dbHelper.getLastTrackName()
					if (lastTrackInDb != track) {
						val currentTime =
						SimpleDateFormat("yyyy-MM-dd HH:mm:ss", Locale.getDefault()).format(Date())
						dbHelper.insertSong(artist, track, currentTime)
					}
					
				} catch (e: Exception) {
					e.printStackTrace()
					updateUIStatus("Помилка парсингу: ${e.message}\nСирі дані:\n$jsonString")
				}
			}
			
			// Helper to update TextView from background threads
			private suspend fun updateUIStatus(message: String) {
				withContext(Dispatchers.Main) {
					tvCurrentStatus.text = message
				}
			}
		}
	\end{minted}
	
	\textbf{activity\_main.xml}
	
	\begin{minted}{xml} 
		<?xml version="1.0" encoding="utf-8"?>
		<androidx.constraintlayout.widget.ConstraintLayout xmlns:android="http://schemas.android.com/apk/res/android"
		xmlns:app="http://schemas.android.com/apk/res-auto"
		android:layout_width="match_parent"
		android:layout_height="match_parent"
		android:padding="16dp">
		
		<TextView
		android:id="@+id/tvCurrentStatus"
		android:layout_width="match_parent"
		android:layout_height="wrap_content"
		android:text="Ініціалізація..."
		android:textAlignment="center"
		android:textSize="18sp"
		app:layout_constraintBottom_toTopOf="@+id/btnViewHistory"
		app:layout_constraintTop_toTopOf="parent"
		app:layout_constraintVertical_chainStyle="packed" />
		
		<Button
		android:id="@+id/btnViewHistory"
		android:layout_width="match_parent"
		android:layout_height="wrap_content"
		android:layout_marginTop="24dp"
		android:text="Переглянути історію"
		app:layout_constraintBottom_toBottomOf="parent"
		app:layout_constraintTop_toBottomOf="@+id/tvCurrentStatus" />
		
		</androidx.constraintlayout.widget.ConstraintLayout>
	\end{minted}
	
	\textbf{HistoryActivity.kt}
	
	\begin{minted}{kotlin} 
		package ua.kpi.lab4
		
		import android.os.Bundle
		import android.widget.TextView
		import androidx.appcompat.app.AppCompatActivity
		
		class HistoryActivity : AppCompatActivity() {
			private lateinit var dbHelper: DatabaseHelper
			
			override fun onCreate(savedInstanceState: Bundle?) {
				super.onCreate(savedInstanceState)
				setContentView(R.layout.activity_history)
				
				dbHelper = DatabaseHelper(this)
				
				val tvDatabaseContent = findViewById<TextView>(R.id.tvDatabaseContent)
				
				// Fetch and display all data from the database
				val allData = dbHelper.getAllSongs()
				tvDatabaseContent.text = allData
			}
		}
	\end{minted}
	
	\textbf{activity\_history.xml}
	
	\begin{minted}{xml} 
		<?xml version="1.0" encoding="utf-8"?>
		<ScrollView xmlns:android="http://schemas.android.com/apk/res/android"
		android:layout_width="match_parent"
		android:layout_height="match_parent"
		android:padding="16dp">
		
		<TextView
		android:id="@+id/tvDatabaseContent"
		android:layout_width="match_parent"
		android:layout_height="wrap_content"
		android:textColor="#000000"
		android:textSize="18sp" />
		
		</ScrollView>
	\end{minted}
\end{document}